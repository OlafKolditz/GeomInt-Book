\section{Model Exercise 3-3 (09): Cycling loading pressure diffusion}
\label{sec:mex09}
%------------------------------------------------------------------------------
\Authors{Patrick Schmidt, Keita Yoshioka, Holger Steeb et al.}
%------------------------------------------------------------------------------
\subsection{Experimental set-up}
%------------------------------------------------------------------------------
\todo[inline]{[UoS]: Please briefly describe/refer experimental set-up}
%------------------------------------------------------------------------------
\subsection{Model approach}
%------------------------------------------------------------------------------
\subsubsection*{Finite element approach: hybrid-dimensional formulation}
A physically sound flow model assuming compressible and viscous fluids in a deformable fracture setting is used to investigate pressure diffusion within fluid-filled fractures. In order to capture local and non-local transient effects a consistent implicit coupling of both domains is necessary to guarantee stability and accuracy. Calculations on non-conformal meshes and two different computational domains motivates a so-called weak coupling scheme implemented in FENICS. A robust numerically scheme is obtained by strongly coupled interface elements implemented in the DUNE framework. 
\subsubsection*{Evaluation of strong and weak coupling scheme}
Implementation of both schemes are firstly evaluated for a single fracture under linear conditions and compared with a reference solution obtained by Biot's formulation. The weak coupling scheme is implemented using a fixed-stress and fixed-strain algorithm providing a physical preconditioning and is mathematically similar to a Richardson type iteration. The system becomes numerically stiff for low fluid compressibilities on a local level since the physics are similar to volumetrically coupled problems. Strong and weak coupling schemes are tested on a single fracture with a length of $100 \, \text{m}$ under a constant fluid pressure of $20 \, \text{kPa}$ with regards of accuracy and stability. 

\begin{table*}[htb]
\centering
\begin{adjustbox}{max width=\textwidth}
%\begin{adjustbox}{angle=90}
\begin{tabular}{llllll}\hline
\rule[1.9ex]{0ex}{1ex}\bf{Quantity} & \bf{Value} & \bf{Unit} & \bf{Quantity} & \bf{Value} & \bf{Unit} \\[1.1ex]\hline
\textbf{\textit{Poroelastic domain $\mathcal{B}^{Pe}$}} &&&&&  \\
intrinsic permeability $k^\mathfrak{s}$ & $1.1\cdot10^{-19}$ & [$m^2$] & min fluid comp. $\beta^f_{min}$  & $4.5\cdot10^{-10}$ & [1/Pa]  \\
max fluid comp. $\beta^f_{max}$  & $4.5\cdot10^{-4}  $ & [1/Pa] & 
sample length $l_{{Pe}}$ & $1.0\cdot10^3$ & [m] \\ 
sample height $h_{{Pe}}$ & $5.0 \cdot 10^2$ & [m] &&&\\
\textbf{\textit{Fracture domain $\mathcal{B}^{{Fr}}$}} &&&&&  \\
min fluid comp. $\beta^f_{min}$  & $4.5\cdot10^{-10} $ & [1/Pa] & max fluid comp. $\beta^f_{max}$  & $4.5\cdot10^{-4}  $ & [1/Pa] \\ 
fracture aperture $\delta_0$ & $5.0\cdot10^{-3}$ & [m] & fracture length $l^{Fr}$ & $1.0 \cdot 10^2$ & [m] \\
pumping pressure $p_0$ & $2.0\cdot10^4$ & [Pa] &  &  & \\
\textbf{\textit{Numerical Parameter}} &&&&&  \\
time step size $\Delta t$ & $1.0\cdot10^{-2}$ & [s] &
fracture discret. in $\mathcal{B}^{Fr}$ $\Delta x^{Fr}_{Fr}$ & $1.0\cdot10^{-1}$ & [m] \\
fracture discret. in $\mathcal{B}^{Pe}$ $\Delta x^{Fr}_{Pe}$ & $1.0\cdot10^{-1}$ & [m] &
evaluation time $t_0$ & $1.0 \cdot 10^1$ & [s] \\
evaluation position $x_0$ & $1.0 \cdot 10^1$ & [m] & error tolerance $\epsilon_{max}$ & $1.0 \cdot 10^{-6}$ & [-] \\
number of DoF & $1.4 \cdot 10^5$ & [-]& & & \\
\hline
\end{tabular}
\end{adjustbox}
\caption{Collection of parameters used for validation of the weak and strong coupling schemes.}
\label{tab:validationparameters}
\end{table*}

\begin{figure}[!ht]
\centering
\includegraphics[width=1\linewidth]{figures/ME9_benchmark_set_up.pdf}
\caption{Discretization and boundary conditions used for the weak coupling (a) and strong coupling scheme
(b) throughout the validation.}
\label{fig:ME9_validation_set_up}
\end{figure}
\begin{figure}
\centering
\includegraphics[width=1\linewidth]{figures/ME9_compressibility_study.pdf}
\caption{Plot of the relative error $\epsilon_{rel}$ for the strong and weak coupling schemes (a) and recommended compressibility range of application for both methods based on the necessary number of iterations $N_{iter}$ of the fixed-strain and fixed-stress scheme for \textbf{L}ow \textbf{C}ompressibilities, \textbf{M}oderate \textbf{C}ompressibilities and \textbf{H}igh \textbf{C}ompressibilities of the fluid (b).}
\label{fig:ME9_validation_compressibility}
\end{figure}

\subsubsection*{Non-Linear Evaluation}
Large deformations in high aspect ratios are reached for small absolute deformation values already and highly influences the fracture flow by local (permeability) and non-local (volumetric) deformation related perturbations. Time dependent pressure/flow stimulation requires a continuous transient change of fracture aperture and deformation dependent pressure diffusion. Validation of the non-linear formulation is using a cylindrical probe with a single fracture under a confining pressure $\sigma_c$ and a harmonic pressure stimulation $p(t)$. In order to study the behaviour the amplitude of stimulation pressure is increased to enforce large deformation changes within a period. Outcome of the validation is a non-linear pressure-flux relationship with a non-constant phase shift.

\begin{table*}[htb]
\centering
\begin{adjustbox}{angle=90}%{max width=\textwidth}
\begin{tabular}{llllll}\hline
\rule[1.9ex]{0ex}{1ex}\bf{Quantity} & \bf{Value} & \bf{Unit} & \bf{Quantity} & \bf{Value} & \bf{Unit} \\[1.1ex]\hline
\textbf{\textit{Rock matrix}} &&&&&  \\
dry frame bulk modulus $K$ & $2.2 \cdot 10^{10}$ & [Pa] & grain bulk modulus $K^\mathfrak{s}$ & $4.6\cdot 10^{10}$ & [Pa] \\
shear modulus $\mu$ & $1.77 \cdot 10^{10}$ & [Pa] & porosity $\phi$ & $1.0\cdot10^{-2}$ & [-]  \\
intrinsic permeability $k^\mathfrak{s}$ & $5.0 \cdot 10^{-19}$  & [$\text{m}^2$] &  
fluid compressibility $\beta^f$  & $4.17 \cdot 10^{-10}$  & [1/Pa] \\
effective bulk modulus $K_{eff}$ & $3.98 \cdot 10^{10}$  & [Pa] &  effective shear modulus $\mu_{eff}$  
& $1.77 \cdot 10^{10}$  & [Pa] \\
\textbf{\textit{Fracture domain}} &&&&&  \\
effective fluid viscosity $\eta^{\mathfrak{f}R}$ & $1.0\cdot10^{-3}$ & [Pa$\cdot$s] & 
initial fracture aperture $\delta_0$ & $2.5 \cdot 10^{-6}$ & [m]\\
equilibrium fracture aperture $\delta_{eq}$ & $4.61 \cdot 10^{-6}$ & [m] &  
fluid compressibility $\beta^f$  & $4.17 \cdot 10^{-10}$  & [1/Pa] \\
\multicolumn{5}{l}{\textbf{\textit{Sample Geometry and Hydraulic Stimulation}}}  \\
sample height $h$ & $7.5\cdot10^{-2}$ & [m] &
sample diameter $d$ & $3.0\cdot10^{-2}$ & [m] \\
borehole diameter $d_b$ & $6.0\cdot10^{-3}$ & [m] &
stimulation period $T$ & $2.0\cdot \pi$ & [sec] \\
equilibrium pressure $p_{eq}$ & $1.5\cdot10^{7}$ & [Pa] &
reference Amplitude $p_A^{lin}$ & $1.0\cdot10^{3}$ & [Pa] \\
minimum Amplitude $p_{A}^{min}$ & $0.5\cdot10^{6}$ & [Pa] &
maximum Amplitude $p_{A}^{max}$ & $3.0\cdot10^{6}$ & [Pa] \\
pressure increment $\Delta p_{A}$ & $0.5\cdot10^{6}$ & [Pa] &
confining pressure $p_{c}$ & $2.0\cdot10^{7}$ & [Pa] \\
\hline
\end{tabular}
\end{adjustbox}
\caption{Collection of parameters used throughout numerical periodic hydraulic stimulation experiment on a sandstone probe.}
\label{tab:non_conformal_parameters}
\end{table*}

\begin{figure}
\centering
\includegraphics[width=1\linewidth]{figures/ME9_experimental_set_up_nl_pq.pdf}
\caption{Numerical set up of cylindrical probe under periodic hydraulic pressure $p(t)$ and
confining pressure $p_c$.}
\label{fig:ME9_validation_set_up1}
\end{figure}
\begin{figure}
\centering
\includegraphics[width=1\linewidth]{figures/ME9_non_linear_p_q.pdf}
\caption{Presentation of $p-q$ comparison via time and hysteresis plot. Flux solutions $q_n$ are
highlighted in green scales for varying pressure amplitudes normalized with respect to its maximum value, the normalized linear flux solution $q_{lin}$ is highlighted in red and the normalized stimulation pressure $p_n(t)$ is highlighted in blue.}
\label{fig:ME9_validation_compressibility1}
\end{figure}

\subsubsection*{Model by variational phase-field model}
The phase-field implementation in OGS is tested under the same condition.
Crack opening in the variational phase-field mode is not direct variable but is a reconstruction from the diffused variables~\ref{fig:ME9_width_VPF}.

\begin{figure}
\centering
\includegraphics[width=1\linewidth]{figures/Keita_ME9_pres.png}
\caption{Pressure simulation from the variational phase-field.}
\label{fig:ME9_pressrue_VPF}
\end{figure}
\begin{figure}
\centering
\includegraphics[width=1\linewidth]{figures/Keita_ME9_width.png}
\caption{Crack opening from the variational phase-field.}
\label{fig:ME9_width_VPF}
\end{figure}

\todo[inline]{[UFZ](KY): Please include VPF results}

%------------------------------------------------------------------------------
\subsection{Results and discussion}
%------------------------------------------------------------------------------

\todo[inline]{[UoS]: Please complete results and discussion}