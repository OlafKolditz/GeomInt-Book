%%==============================================================================
%  Model exercise Desiccation under in-situ conditions
%  HS, 05/04/2019
%%==============================================================================
\section[MEX 1-3: Desiccation under in-situ conditions]{Model Exercise 1-3: Desiccation under in-situ conditions}
\label{sec:mex12}
\index{desiccation process}
%------------------------------------------------------------------------------
\Authors{Holger Steeb}
%------------------------------------------------------------------------------
We focus on in-situ (image-based) characterization of desiccation processes of sandy-clay samples
under confined conditions. A cylindrical clay core with diameter $D=30$ mm and height 
$H=60$ mm is embedded in a hollow PEEK (Polyether ether ketone) tube. 
The sample is prepared with a drill hole of $d=5$ mm.
Further, the sample is firstly sealed with a Teflon shrink tube. Secondly the gap between the sample and the PEEK tube is filled with an epoxy resin.
Epoxy and PEEK have low X-Ray absorption properties. Through the embedding, radial deformations of the clay sample are preventing. The top and bottom parts of the samples are (hydraulically) sealed with end-caps. The end-caps allow for axial deformations (should we measure them?) occurring during the shrinkage process. The drill hole is ``open''  allowing for water release/uptake.
The prepared sample with in-situ humidity (fully/partially saturation?) is placed in an environmental chamber (Anton Paar, CDT 100, Peltier heated). Temperature and humidity can be controlled (MHG 100 humidity controller, ProUmid). We plan to plug in a small van which allows for controlled humidity convection in the drill hole. 

Drying/desiccation will be image-based characterized within the open XRCT scanner in Stuttgart. We are aiming for a spatial resolution of $30 / 2500$ = 12 $\mu$m / voxel. XRCT scans will be performed for equilibrium states at certain humidity conditions (at ambient temperature of $\Theta = 20^o$ C).

\todo[inline]{[UoS](HS): Adding a figure (experimental set-up/device)?}
