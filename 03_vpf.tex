\subsection{PFM - Variational Phase-Field Method}

Phase-field models have become one of the most established numerical techniques for fracturing simulation in the last two decades. 
The approach was first introduced by Bourdin et al.~\cite{Bourdin2000} as a numerical implementation method to the variational approach to fractures proposed by Francfort et al.~\cite{Francfort1998}. 
Since the initial inception of the method, models for brittle and cohesive fracture have been further studied by many others~\cite{Bourdin2008, Hakim2009, Amor2009,  Kuhn2010, Verhoosel2010,  Pham2011, Vignollet2014,  Ambati2015, Marigo2016, Wu2017, Tanne2018,  Sargado2018} including advanced numerical solution schemes~\cite{Gerasimov2016, Farrell2017}.
Lately, its application ranges from ductile fracturing~\cite{Ambati2015, Miehe2015_thermo,Kuhn2016,Alessi2017} to fatigue~\cite{Alessi2018,Seiler2018}, desiccation fracture~\cite{Maurini2013, Cajuhi2017}, and dynamic fracturing~\cite{Bourdin2011, Borden2012,Hofacker2012, Schluter2014, Li2016}.
\index{ductile fracturing}
\index{fatigue}
\index{dynamic fracturing}
\index{desiccation fracture}

\subsubsection*{Regularisation of the total energy functional}
In the variational approach to fractures in~\cite{Francfort1998}, the total energy in the system is defined as the sum of the elastic strain energy, the potential of the external forces and the surface energy as:
\index{total energy functional}
\index{elastic strain energy}
\begin{equation}
\label{eq:defE}
E(\mathbf{u},\Gamma_\mrm{c}) := \int_{\Omega\setminus \Gamma_\mrm{c}} \psi (\mathbf{u}) \, \mathd \Omega -\int _{\partial_{\mrm{N}} \Omega} \bar{\mbf{t}}\cdot \mathbf{u} \, \mathd \Gamma - \int_{\Omega\setminus \Gamma_\mrm{c}} \varrho \mathbf{b}\cdot \mathbf{u} \, \mathd \Omega + G_\mrm{c} \int_{\Gamma_\mrm{c}} \, \mathd \Gamma,
\end{equation}
where $\psi$ is the strain energy density, $\bar{\mbf{t}}$ is the boundary traction force, $\mathbf{u}$ is the displacement, $\varrho$ is the mass density of the porous
medium, $\mathbf{b}$ is the applied specific body force, and $G_\mrm{c}$ is the critical energy release rate. 
For hydraulic fracturing, the work done by the fluid pressure within the fracture $p$ needs to be added to the energy, and the total energy becomes:
\begin{equation}
\label{eq:defEwithP}
E(\mathbf{u},\Gamma_\mrm{c}) := \int_{\Omega\setminus \Gamma_\mrm{c}} \psi 
(\mathbf{u}) \, \mathd \Omega 
-\int _{\partial_{\mrm{N}} \Omega} \bar{\mbf{t}}\cdot \mathbf{u} \, \mathd \Gamma
- \int_{\Omega\setminus \Gamma_\mrm{c}} \varrho \mathbf{b}\cdot \mathbf{u} \, \mathd \Omega 
+ G_\mrm{c} \int_{\Gamma_\mrm{c}} \, \mathd \Gamma \\
+ \int_{\Gamma_\mrm{c}} p \jump{\mbf{u}} \cdot \mbf{n}_\Gamma \, \mathd \Gamma,
\end{equation}
where $\mbf{n}_\Gamma$ is the normal direction to the crack set $\Gamma$. 
Following Bourdin et al.~\cite{Bourdin2012}, with introduction of a phase-field order variable (or damage variable) $d$~\eqref{eq:defEwithP} is regularised as:
\index{phase-field order variable}
\index{damage variable}
\begin{equation}
\label{eq:regEwithregP}
\begin{split}
E(\mathbf{u},d,p) =   \int_{\Omega} (1-d)^{2} \psi(\mathbf{u}) \mathd \Omega -\int _{\partial_{\mrm{N}} \Omega} \bar{\mathbf{t}}\cdot \mathbf{u} \, \textbf{ds} - \int_\Omega \varrho \mbf{b}\cdot \mathbf{u} \, \mathd \Omega \\
+ \frac{G_\mrm{c}}{4c_w} \int_\Omega \left( \frac{w(d)}{\ell} + \ell \left| \nabla d \right|^{2} \right) \, \mathd \Omega + \int_\Omega p \mathbf{u} \cdot \nabla d \, \mathd \Omega. 
\end{split}
\end{equation}
where $d$ is the damage variable that equals 0 when undamaged and 1 for a fully damaged state, $c_w$ is a normalization parameter defined as $c_w:=\int_{0}^{1}\sqrt{w(s)}\mathd s$, $\ell$ is a regularisation length, and $w(d)$ is the dissipative energy function.
Various possible dissipative energy functions $w(d)$ are discussed in \cite{Marigo2016}. 
The most widely used function is a quadratic form~\cite{Bourdin2000, Kuhn2010, Miehe2010a, Klinsmann2015} $w(d) = d^2$.
The variational phase-field model with this choice of the dissipative function is called \ATtwo{} model in~\cite{Bourdin2014}, and we follow this terminology in this report.
Another choice for $w(d)$ is a linear form $w(d) = d$ and is known as \ATone{}.
A thorough study comparing these two models in terms of fracture nucleation and propagation can be found in~\cite{Tanne2018}.
Note that in arriving at~\eqref{eq:regEwithregP}, the work done by the fracture pressure in the crack is approximated as $\int_\Omega p \mathbf{u} \cdot \nabla d\, \mathd \Omega$. 
\index{dissipative energy function}
\index{fracture nucleation}

\subsubsection*{Numerical implementation}
The solution of~\eqref{eq:regEwithregP} follows the alternate minimisation scheme introduced in \cite{Bourdin2000} with respect to $\mathbf{u}$ and $d$ given a time-evolving $p$. 
Thus, the minimisation problem can be stated as
\begin{equation}
\label{eq:min_prob}
(\mathbf{u}, d)^* = \underset{\small\begin{cases} 
	\mathbf{u} \in H^1 \\ 
	d \in H^1, d^t \subset d^{t+\Delta t} \\
	\end{cases}}{\mrm{arg}\,\min \tilde{E}(\mathbf{u}, d, p)},
\end{equation}
The first variation of the energy functional with respect to $\mathbf{u}$ is
\begin{align}
\label{eq:WeakForm_u}
\delta E(\mathbf{u},d,p; \delta \mathbf{u}) &=   
\frac{1}{2} \int_{\Omega} (1-d)^{2} \mbf{C}(\mbfs{\mathbf{u}}))\dcdot\mbfs{\epsilon}(\delta \mathbf{u})  
	\, \mathd \Omega - \\
	&- \int _{\partial_{\mrm{N}} \Omega} \mathbf{t}\cdot \delta \mathbf{u} \, \mathd \Gamma  
	-  \int_\Omega \varrho \mbf{b}\cdot \delta \mathbf{u} \, \mathd \Omega 	
	+  \int_\Omega p \delta \mathbf{u} \cdot \nabla d \, \mathd \Omega, \\
\end{align}
where $\mbf{C}$ is the constitutive relation that satisfies $\psi = \mbf{C}(\mbfs{\epsilon}):\mbfs{\epsilon}/2$. 
The first variation of the energy functional with respect to $d$ of \ATone{} is: 

\begin{align}
\label{eq:WeakForm_d_AT1}
\begin{split}
%\begin{equation}
\delta E(\mathbf{u},d,p; \delta d) &=   
-\int_{\Omega} d \delta d \mbf{C}(\mbfs{\epsilon}(\mathbf{u})) \dcdot \mbfs{\epsilon} ( \mathbf{u}) \mathd \Omega 
\\
&+ \frac{3G_\mrm{c}}{8} \int_\Omega \left( \frac{\delta d}{\ell} +2 \ell  \nabla d \cdot \nabla \delta d \right) \, \mathd \Omega
+  \int_\Omega p \mathbf{u} \cdot \nabla \delta d \, \mathd \Omega, \\
%\end{equation}
\end{split}
\end{align}

and \ATtwo{} is

\begin{align}
\label{eq:WeakForm_d_AT2}
\begin{split}
%\begin{equation}
\delta E (\mathbf{u},d,p; \delta d) &=   
-\int_{\Omega} d \delta d \mbf{C}(\mbfs{\epsilon}(\mathbf{u})) \dcdot \mbfs{\epsilon}( \mathbf{u}) \mathd \Omega 
\\
&+ G_\mrm{c} \int_\Omega \left( \frac{d}{\ell} \delta d + \ell  \nabla d \cdot \nabla \delta d \right) \, \mathd \Omega
+  \int_\Omega p \mathbf{u} \cdot \nabla \delta d \, \mathd \Omega. \\
%\end{equation}
\end{split}
\end{align}

It should be noted that the solution of~\eqref{eq:WeakForm_d_AT1} or~\eqref{eq:WeakForm_d_AT2} for $d$ is not bounded in $[1,0]$.
Therefore, the solution scheme would  require enforcement of the variational inequality ($ 0 \leq d \leq 1$). 
For this study, a variational inequality non-linear solver of the PETSc library \cite{petsc-web-page, petsc-user-ref} has been applied to fullfill the bound of $d$.