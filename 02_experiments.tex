\chapter{Experimental Platform}
\label{cha:exp}

In order to investigate the barrier rocks, such as saltstone, claystone and crystalline, response under the coupled thermo-hydro-mechanical (THM) processes, a series of laboratory and field tests in the scope of the GeomInt project are carried out. Initially, an introduction to the tested barrier rocks specifications and characteristics are provided. Next, the Opalinus claystone and saltstone responses under the coupled THM processes are investigated. The three-point bending, Brazilian disk and true triaxial tests are used to determine the fracture toughness, splitting strength and change of the mechanical and thermal properties under the coupled loadings, respectively. It is shown that the embedded layering orientation in the claystone samples has a significant effect on materials strength, deformation and frack paths. After determining the general material properties, distinct laboratory tests, which are chosen based on the unique purpose of the GeomInt project, are performed to analyze the shrinkage and swelling behavior of the claystone samples under free or in-situ loading conditions.  The results again indicate the effect of the anisotropy on the swelling and shrinkage direction of the claystone material. The fluid or gas driven percolation tests on Opalinus claystone and saltstone samples explore the change of the permeability, the required fracking pressure and the principle stress dependent frack paths. Additionally, the healing and frack closure characteristic of the barrier rocks are investigated. Moreover, the effect of the compressibility on the flow flux and drop of the reservoirs pressure when using brine or gas as fracking medium is presented. The direct shear test on the crystalline gives an insight of the shear characteristics of the joints under the constant normal load (CNL) boundary condition. Additionally, the constant normal stiffness (CNS) test with the additional of extra load to the CNL procedure is performed. Specifically, for the crystalline samples, the diffusion and time-dependence states of the fractures under the cyclic loading conditions are characterized. Eventually, the flow and pressure dependency on the deformations of the frack volumes and paths are studied.

\begin{figure}[!ht]
\centering
\includegraphics[width=10cm,height=12cm]{figures/Amir_Experiment.png}
\caption{The conducted laboratory tests in the scope of GeomInt research project}
\label{fig:Amir_Experiment}
\end{figure} 



%\begin{figure}[h!]
%\centering
%\includegraphics[width=1\textwidth]{figures/geomint-exp-overview}
%\label{fig:10}
%\caption{Platzhalter}
%\end{figure}
%===================================================================================================
\section{Rock Material Properties}
%-----------------------------------------------------------
\subsection{Opalinus Claystone from Mont-Terri, Switzerland}

\Authors{Tilo Kneuker, Bernhard Vowinckel, Gesa Ziefle, Jobst Maßmann (BGR)}

\subsubsection{The Mont Terri Rock laboratory}\label{sec:mont_terri}

Opalinus claystone is a very promising hostrock for the safe disposal of heat emitting nuclear waste. This type of hostrock has been investigated in the Mont Terri Rock laboratory for more than 24 years. The Mont Terri rock laboratory is a facility to conduct research in the deep geological underground at in-situ scale, such as the safe deposition of radioactive waste, where the local host rock is Opalinus Clay. The rock laboratory is located within the Jura Mountain fold belt. The development of the Jura fold belt began in the Middle Miocene around 12 million years ago, which was constrained by the first occurrence of overthrusted and folded molasse sediments \cite{bolliger1993}. The overthrust of the frontal fold and thrust belt over the allochthonous foreland (Tabular Jura) occurred ca. 10.5 million years ago \cite{becker2000}. More specifically, the Mont Terri rock laboratory is located in the southeast dipping fold limb of the NW-vergent Mont Terri anticline \cite{nussbaum2011}. The total amount of shortening of the anticlinal structure is approximately 2.1 km \cite{freivogel2003}. During the folding process, the northwestern fold limb of the Mont Terri anticlinal structure was sheared-off and now lies on top of the Tabular Jura (Figure \ref{fig:bgr_mt_sideview}). The Mont Terri anticlinal structure developed in a special structural setting at the intersection between the frontal part of the Jura fold belt (main shortening direction NW-SE) and the roughly N-S oriented structural elements of the Rhine-Bresse-graben transfer zone \cite{nussbaum2011}.

\begin{figure}[!ht]
\centering
\includegraphics[width=1\textwidth]{./figures/bgr_mont_terri_side_view.png}
\caption{Geological cross section along the motorway tunnel through the Mont Terri anticline. From: Kaufhold et al. (2016) \cite{kaufhold2016}, based on Freivogel \& Huggenberger (2003) \cite{freivogel2003}.}
\label{fig:bgr_mt_sideview}
\end{figure}

The Mont Terri rock laboratory branches off from the security gallery of the motorway tunnel near the town of St. Ursanne (NW Switzerland). The rock laboratory is located entirely within the Middle Jurassic Opalinus Clay formation. The thickness of the Opalinus Clay in the rock laboratory is around 130 m \cite{hostettler2018} and the layers are dipping with ca. 40° towards SE. The depth below ground varies between 230 m and 320 m, depending on the topography \cite{heitzmann2001}. Since 1996, a total of 1400 m of galleries and niches have been excavated in the Mont Terri rock laboratory (Figure \ref{fig:bgr_mt_topview}). The Mont Terri rock laboratory is a generic scientific research laboratory. At Mont Terri, there will be no storage of radioactive waste.

\begin{figure}[!ht]
\centering
\includegraphics[width=1\textwidth]{./figures/bgr_mont_terri_top_view.jpg}
\caption{Geological map of the rock laboratory with all in-situ experiments relevant for the GeomInt project including the locations of the two AD-boreholes (BAD-1, BAD-2), the drilling for the FS-experiment (BFS-1 to 3), the EZ-B niche for the CD/LP experiment and its successor experiment CD-A in Gallery 18. The different facies types of the Opalinus Clay can be recognized by the different shades of brown and yellow (map modified from Mont Terri Consortium, swisstopo).}
\label{fig:bgr_mt_topview}
\end{figure}

The Opalinus Clay in the rock laboratory is composed of a dark gray claystone that was named after the ammonite species Leioceras opalinum. This claystone formation was deposited during the period of the Toarcian/Aalenian, at an age of approximately 174 million years. The Opalinus Clay is exposed along the rim of the Swabian and Franconian Alp in Germany and stretches into northern Switzerland \cite{einsele1983}. The Opalinus Clay was deposited in a shallow-marine, epicontinental milieu in the area of the storm wave base at approximately 20 m to 50 m water depth \cite{wetzel2003}. Coarser siliciclastic components are of detritical origin. Potential sources for the detritic components are the areas of the Bohemian Massif and the Vindelician Landmass \cite{wetzel2003}. During Cretaceous burial, the Opalinus Clay experienced maximum palaeo temperatures of $75^{\circ}$~C to $90^{\circ}$~C \cite{bossart2008} at a maximum burial depth of 1.35 km. The Opalinus Clay at the Mont Terri site is underlain by  marls of Upper Toarcian age and overlain by limestones (Bajocian), some of which act as karst aquifer \cite{pearson2003}.

The Opalinus Clay at the Mont Terri rock laboratory can be subdivided into three main facies types \cite{bossart2008}. First, the shaly facies occupies the largest area of the rock laboratory (Figure \ref{fig:bgr_mt_topview}). It dominates the lower part of the Opalinus Clay formation. It consists of mica-bearing clay and marly shales as well as flasery, marly layers characterized by bioturbation. The clay-rich facies occurring in the upper part of the profile contains a higher volumetric content of quartz grains. Second, the sandy facies occurs in the middle and upper part of the profile (lower and upper sandy facies). It includes medium gray marly claystones with intercalated, bioturbated marly layers or lenticular, gray sandy limestones and pale sand layers of approximately 1-10 mm thickness that include pyrite as well. Third, a carbonate-rich sandy facies of 5 m thickness occurs in the middle part of the rock formation. It consists of calcareous sandstones with intercalated bioturbated limestone layers, which show a relatively high proportion of detritic quartz and white mica. The different facies types of the Opalinus Clay can be attributed to varying sedimentation conditions in a shallow marine environment (like variations in depth and current directions). The carbonate-rich facies is typical for the Jura region in western Switzerland and it doe not occure in the siting regions.

The mineralogical composition of the Opalinus Clay was examined by Traber \& Blaser (2013) \cite{traber2013} for several locations in northern Switzerland. For the clay-rich facies, the clay mineral content varies between 40 wt\% and 75 wt\%. The clay minerals determined by Traber \& Blaser (2013) \cite{traber2013} include illite, kaolinite and smectite-illite mixed layer minerals. According to Traber \& Blaser (2013) \cite{traber2013}, the proportion of swellable clay minerals is around 10 wt\%. Detritic components such as quartz and feldspars typically make up to 20 wt\% of the investigated samples. The carbonate content (calcite and dolomite) is around 20 wt\%. The sandy facies of the Opalinus Clay is composed of up to 40 wt\% clay minerals and ca. 30 wt\% quartz; it shows a lower amount of clay minerals in favor of a higher quartz content, compared to the shaly facies \cite{heitzmann2001}. 

The BGR has been involved in a number of campaigns to study in-situ conditions of clay rock, of which the following four are particularly noteworthy within the context of the GeomInt-Project. First, the CD/LP experiment investigates the long-term cyclic deformation (CD) due to seasonally induced cyclic swelling and shrinkage in a niche of the rock laboratory. In addition, a follow-up project, the CD-A experiment, has been prepared in recent years, to distinguish between deformation processes due to stress redistribution and seasonal variations in air humidity that cause saturation (swelling) and desaturation (shrinkage) of the rock and stress redistribution alone. To this end, two identical niches were excavated, one sealed towards the gallery and with a high humidity inside to minimize desaturation and one open to the general air circulation of the rock laboratory. The measurement campaign was started in October 2019 \cite{ziefle2019}. Third, the AD-experiment intends to provide an improved process understanding for the experimental-numerical analysis of discontinuities. Finally, the Fault Slip (FS) – experiment addresses the fault reactivation due to pressure-induced percolation in a low-permeability, large-scale discontinuity in the Mont Terri rock laboratory. The AD is directly relevant to the numerical and experimental investigations presented in Sections \ref{sec:mex05} - \ref{sec:mex12}, because the rock material used in these experiments were drilling samples from this experiment. Hence, we provide a brief overview for the campaign in the following. 

\subsubsection{The CD/LP Experiment in the Mont Terri Rock laboratory}

The Mont Terri Rock laboratory in Switzerland hosts a multitude of in-situ experiments that investigate the response of Opalinus claystone to various geotechnical applications. An overview of the rock laboratory is given in Section \ref{sec:mont_terri}. In particular, the CD (Cyclic Deformation) experiment has been a valuable site to gather experimental data at the in-situ scale to investigate the hydraulic-mechanial coupling induced by swelling and shrinking of Opalinus claystone due to cyclic variations of air humidity. Section \ref{sec:mex10} focuses on the numerical investigation of these processes. Here, we provide a brief overview of the experimental CD campaign at the Mont Terri Rock Laboratory. 

The experiment itself is located in the EZ-B niche (Figure \ref{fig:bgr_mt_topview}). The experiment has been conducted for more than 13 years to provide information on the swelling and shrinkage behavior of Opalinus Clay in the Mont Terri rock laboratory. The idea was to analyze a niche that is not covered by shotcrete. Instead, the clay rock remains in direct contact with the atmospheric conditions of the main gallery for the entire time. Consequently, the swelling and shrinkage is induced by changes in temperature and relative humidity, which can decrease to values as low as 65\% in the winter and reaches values of up to 100\% in the summer. 

\begin{figure}[!ht]
\centering
\includegraphics[width=1\textwidth]{./figures/bgr_CD_experiment.jpg}
\caption{The EZ-B niche in the Mont Terri rock laboratory, where the CD/LP experiment has been conducted since 2006 (photo: Mont Terri consortium, swisstopo).}
\label{fig:bgr_CD_experiment}
\end{figure}

A special focus of this experiment was to investigate the long-term impact of these seasonal variations on the temporal evolution of the cracks that occur during the excavation process and make up the Excavation Damaged Zone (EDZ). To this end, the EZ-B niche was excavated in the years 2004/2005 (Figure \ref{fig:bgr_CD_experiment}). Subsequently, the niche was equipped with a comprehensive set of measurement devices to record the evolution of temperature, water content, convergence of the niche and crack development at the tunnel walls over time. This measurement campaign was started in 2006 and has been continued until today to investigate long-term effects. Note that the experiment was transferred into the LP-A experiment to explicitly focus on the long-term monitoring of pore pressure. The CD/LP experiment under in-situ conditions was supplemented by laboratory experiments with drill cores to determine hydraulic-mechanical properties of the clay rock, such as porosity, grain density, etc. \cite{matray2013}. 

Characteristic macroscopic cracks on the tunnel walls have been monitored and the field data of the crack opening show a good correlation with the seasonal variation of temperature and humidity. The cyclic deformation of the crack opening yields a re-occurring compression perpendicular to the crack during summer, which typically is a time of high relative humidity and, hence, the swelling causes an increase of rock volume \cite{jaeggi2012}. This characteristic behavior of swelling and shrinking was successfully reproduced by means of hydraulic-mechanically coupled numerical simulations \cite{ziefle2018}, which provides valuable benchmark data for future investigations of the cyclic deformation of clay rock.

\subsubsection{The AD-Experiment}

The aim of the experiment is to provide core samples from the sandy facies of the Opalinus Clay as a typical example of an argillaceous host rock for the safe disposal of nuclear waste. These samples were used for experimental-numerical analysis in the framework of the GeomInt project. Additionally, a geological characterization of the cores and seismic (Interval Velocity Measurements - IVM) and geolelectrical measurements (Electrical Resistivity Tomography - ERT) in the boreholes were performed. The results of the experimental campaign yield a valuable description of the sandy facies in addition to the well characterized shaly facies of the Opalinus Clay \cite{bossart2008,jahn2016}. 

From a geological perspective, the AD experiment gave opportunity to study the lower sandy facies of the Opalinus Clay at the Mont Terri rock laboratory in detail. The two fully cored boreholes BAD-1 and BAD-2 with a diameter of 131 mm (yielding samples of 101 mm diameter) were drilled parallel and perpendicular to the sedimentary bedding, respectively. The 15.35 m long horizontal borehole BAD-1 was drilled from 7th-10th of July 2018 by the BGR. It is located entirely in the lower sandy facies. The geological mapping was performed by swisstopo \cite{galletti2019}. The core material of BAD-1 was entirely sampled for laboratory experiments performed by the Christian-Albrechts University of Kiel (Germany) and the Institute of Geomechanics (IfG) Leipzig (Germany). The core samples were conditioned in aluminum foil and pressurized in special nitrogen-filled BGR-liners (autoclaves) to avoid further alteration. 

The BAD-2 borehole has a length of 27.0 m. It was drilled by the BGR team from 9th-17th of May 2018. The borehole is oriented perpendicular to the bedding (with a dip of 43°), thus crossing several facies types of the Opalinus Clay. The geological mapping by Swisstopo reported by Galletti \& Jaeggi (2019) \cite{galletti2019} revealed the following sequence with varying quantities of quartz, carbonates (cements and fossils) and clay minerals: 

\begin{list}{-}{\leftmargin=1em \itemindent=0em \itemsep=0em}
\item 0.0 m to 4.75 m drillcore depth: upper shaly facies,
\item 4.75 m to 19.4 m drillcore depth: lower sandy facies,
\item 19.4 m to 24.57 m drillcore depth: carbonate-rich sandy-facies, 
\item 24.57 m to 27.0 m drillcore depth: lower shaly facies.
\end{list}

This subdivision is confirmed by petrographic-structural studies and geoelectrical resistivity measurements (ERT) performed by the BGR. The BAD-2 drillcores were sampled from 4.0 m to 14.8 m (lower sandy facies) for laboratory experiments by the Universities of Kiel and IFG Leipzig. Following the procedure employed for the BAD-1 drilling, the core material was conditioned in aluminum foil and pressurized in special nitrogen-filled BGR-liners (autoclaves). The drillcore material from the intervals between 0.0 m to 4.0 m and 14.8 m to 27 m, including the transition towards the underlying carbonate-rich facies, are stored at the BGR facility in Hannover (Germany) for further geological characterization. The first results revealed a good core quality and confirm a rather uniform appearance of the sampled profile inside the lower sandy facies, the drillcore material is thus suitable for the planned experiments (cf. Figure \ref{fig:bgr_AD_experiment}).

\begin{figure}[!ht]
\centering
\includegraphics[width=1\textwidth]{./figures/bgr_AD_experiment.jpg}
\caption{Schematic profile of borehole BAD-2 as marked in Figure \ref{fig:bgr_mt_topview} (left), macrostructural (on drillcore scale) and microstructural features (on thin section scale) of the different facies types of Opalinus Clay encountered in the BAD-2 borehole.}
\label{fig:bgr_AD_experiment}
\end{figure}
\clearpage
%---------------------------------------------------------------------------------------------------
\subsection{Rock Salt Samples}
\label{subsec:salt}
\Authors{Mathias Nest, Dirk Naumann (IfG)}

\subsubsection{The Springen in-situ laboratory}
\label{sec:springen}

The large-scale test site Merkers benefits from the unique mining situation in the bedded salt mass of the Werra salt formation (z1, Zechstein sequence) where two potash seams were mined in a room-and-pillar system at 275 m (1st floor, potash seam ''Hessen'', z1KH) and 360 m (2nd floor, potash seam ''Thüringen'', z1KTh) depth, respectively. Fig. \ref{fig:springenlab} shows the preparation of an experiment on the 2nd seam. The evaporite rocks of the Zechstein formation were laid down during the Permian period around 250 million years ago. The intact mineral deposit was locally disturbed between 14 and 25 million years ago by tertiary volcanism, leading to the mutation of some potash salts to sylvinite, and the creation of pockets of CO$_2$ under high pressure.

\begin{figure}[ht!]
\centering
\includegraphics[width=0.85\textwidth]{figures/springen2ndseam.png}
\caption{Preparation of experiments on the second seam.}
\label{fig:springenlab}
\end{figure}

The pressurized tests are conducted between the two potash seams in the very homogeneous Middle Werra rocksalt (z1Na). It consists mostly of very pure halite layers intersected by thin anhydrite lines or bands of rocksalt with finely distributed anhydrite accessories indicating the sedimentary bedding. 

Because the test results depend mainly on the acting stress field, i.e. the minimal stress distribution in the rock mass around the test area, it has been measured and  characterized by hydro-frac measurements, and is thus well-known. The minimal stresses in the contour increase with progressive distances from the underground openings until reaching a constant value in a depth of around 15 m. The measured value of an undisturbed stress state of around 8 MPa corresponds fairly well to calculated lithostatic stresses of 7.8 to 8.8 MPa. 

The main facility is a large borehole of nearly vertical 60 m height and 1.3 m diameter. It was drilled upwards from the second floor, ending about 20 m beneath the first floor. For access to the later sealed volume an 85 mm pilot hole has been drilled from the upper floor. The borehole was closed by a 20 m high MgO-concrete plug. 

For monitoring of micro-seismic events, e.g. due to creation of an excavated damage zone around underground openings or fluid flow driven damage, a highly sensitive AE-network was installed in observation boreholes, which were drilled parallel to the main borehole. This network has constantly been updated and extended over the past years. Signals of magnitude M $<$ -4 can be detected in a frequency range from 1 kHz to 200 kHz. This corresponds to intergranular microcracking on grain boundaries on a millimeter- to centimeter-scale. 

\subsubsection{Rock salt laboratory}

The following conditions and equipment are available in the IfG labs for rock mechanical laboratory investigations:

\begin{list}{-}{\leftmargin=1em \itemindent=0em \itemsep=0.1em}
\item Climate-controlled rooms for storage of specimens at conditions which correspond to those present in situ
\item Laboratory for mineral-petrographic examinations, density and moisture determination, ultra-sound measurements and 
photographic documentation
\item Workshop for the preparation of specimens in high precision according to testing requirements (rock saws, lathes etc.)
\item Test laboratory containing various servo-controlled hydraulic testing machines for conducting investigations on 
rock mechanics in accordance with the up-to-date state of research and development (see Figure \ref{fig:ifglabph1}).
\end{list} 

\begin{figure}[!ht]
\centering
\includegraphics[width=1\textwidth]{./figures/ifg-lab-photo1-v2.png}
\caption{View inside the rock mechanical lab with various servo-controlled hydraulic testing machines.}
\label{fig:ifglabph1}
\end{figure}

It has to be mentioned that some of the applied IfG test procedures have been developed for the requirements of the specific IfG-material laws. 
But generally they are in accordance to ASTM and ISRM standards resp. equivalent descriptions, e.g.:

\begin{list}{-}{\leftmargin=1em \itemindent=0em \itemsep=0.1em}
\item ASTM D 4543-85 Standard Practice for Preparing Rock Core Specimens and Determining Dimensional and Shape Tolerances
\item ASTM D 2845-05 Standard Test Method for Laboratory Determination of Pulse Velocities and Ultrasonic Elastic Constants of Rock
\item ASTM D 2664-86 Standard Test for Triaxial Compressive Strength of Un\-drained Rock Core Specimens without Pore Pressure Measurements 
\item DGEG (1979):   Empfehlung Nr. 2 des AK 19 der DGEG (Dreiaxiale Druckversuche).
\item ASTM D4406-04 Standard Test Method for Creep of Cylindrical Rock Core Specimens in Triaxial Compression
\item ASTM D7070-04 Standard Test Method for Creep of Rock Core Under Constant Stress and Temperature
\item ISRM: Suggested Methods for Determining o the Uniaxial Compressive Strength and Deformability of Rock Materials
\item ISRM: Suggested Methods : Part 2 : 2007 - Unconfined Compressive Strength with Young's Modulus and Poisson's Ratio
\item ISRM: - Suggested Methods : Part 2 : Received 1983 - Strength of Rock Material in Triaxial Compression
\end{list}

\subsubsection{Sample characterization and pre-investigation}

Several preliminary investigations are usually done before lab testing. After preparing the cylindric samples (cutting with a rock saw and smoothing the samples with a lathe) in the IfG labs, their density is determined by measuring the geometrical dimensions and the mass. Concerning the quality of these parameters an accuracy in length determination is $<$ 0.01 mm, those in mass determination is $<$ 0.01 g. 

Additionally, ultrasonic investigations are carried out to check integrity, homogeneity and isotropy of the samples. The ultrasonic pulse measurement system used for transmission of the rock specimens consists of two transducer sets for P-waves and S-waves and a receiver system for generating and evaluating the ultrasonic signals. The specimen is placed in physical contact between two piezoelectric transducers, one acts as a driver and the other one acts as a receiver. The transit time of the mechanical pulse to pass through the specimen is used to determine elastic wave velocity.

For samples where both P- and S-waves were measured in axial sample direction the elastic constants are obtained from density ($\rho$) and the ultrasonic velocities ($v_p$, $v_s$) using the following expressions derived from the theory of elasticity for homogeneous, isotropic solids:

\begin{equation}
\label{eq:YoungsModulus_Ultrasonic}
E_{dyn} = \frac{\rho v_s^2(3v_p^2-4v_s^2)}{v_p^2-v_s^2} 
\end{equation}
\begin{equation}
\label{eq:PoissonsRatio_Ultrasonic}
\nu_{dyn} = \frac{v_p^2-2v_s^2}{2(v_p^2-v_s^2)}
\end{equation}

Dynamic elastic parameters of the various rock portions determined on the base of sonic wave velocity at room temperature are in a wide range representing typical values for the various materials as known from other locations. 


\clearpage
%---------------------------------------------------------------------------------------------------
\subsection{Crystalline Rock Samples}
\label{subsec:crystalline}
\Authors{Thomas Fr\"uhwirt, Daniel P\"otschke}

\subsubsection{URL Reiche Zeche Freiberg}
\index{URL Reiche Zeche}

\begin{wrapfigure}{l}{6cm}
\centering
\includegraphics[width=5.9cm]{figures/reiche-zeche.jpg}
\caption{The Reiche Zeche in Freiberg, \cite{ReicheZechePicture}.}
\end{wrapfigure}
The Reiche Zeche mine is located north-east of the city center of Freiberg, Saxony. It operated as a silver mine for several centuries. It became a part of the TU Freiberg 1919 when silver mining definitely was no longer profitable. Nowadays the mine is used as an underground research laboratory (URL) and for teaching purposes. The roughly $4\,\text{km}^2$ sized area is well documented in terms of geology, mineralogy, geophysics and geometry. Draining of the mine is done using the "Rothsch\"onberger Stollen (tunnel)". A detailed overview about the history of the Reiche Zeche can be found in \cite{ReicheZecheHistory}.
%
Due to its long history, the development of new technologies and the importance for the development of the whole region the mining sites and the associated infrastructure are listed as UNESCO world heritage site since 2019.

Current projects are for example dealing with bio-leaching or complex experiments which study the influence of hydro-fracking experiments on the stress state (STIMTEC project).
\index{STIMTEC project}

The Reiche Zeche is equipped with an underground railway system, installed electricity, water and air pressure. The experienced staff and the nearby located mining agency help to successfully conduct experiments in about 150 m depth. 

\subsubsection{Rock material used in the direct shear tests}

\begin{figure}
\begin{subfigure}[c]{0.48\textwidth}
\includegraphics[width=0.99\textwidth]{./figures/ExpRockGranite.JPG}
\subcaption{Granite sample}
\label{fig:RockGranite}
\end{subfigure}
\begin{subfigure}[c]{0.48\textwidth}
\includegraphics[width=0.99\textwidth]{./figures/ExpRockBasalt.jpg}
\subcaption{Basalt sample}
\label{fig:RockBasalt}
\end{subfigure}
\caption{Crystalline rock samples under investigation}
\end{figure}

Two different crystalline rock \index{crystalline rock} types are used. Granite is a coarse-grained intrusive igneous rock. The grains are on the millimetre to centimetre scale, see Fig. \ref{fig:RockGranite}. The typical main minerals of granites are quartz, feldspar and plagioclase. The used granite origins from Kirchberg, Saxony.
%
Basalt is an fine-grained extrusive igneous rock. It is rich of plagioclase. See Fig. \ref{fig:RockBasalt}. Its origin is V\"olkershausen, Thuringia.
%
Lab tests to evaluate basic rock parameters of the intact rock material have been conducted. The values of the granite and basalt used in the experiments can be found in Tab. \ref{table:MEX7_rockParam}.

\begin{table}[!ht]
\begin{center}
\begin{tabular}{l c r r r}
Parameter & Symbol & Granite & Basalt & Unit\\
\hline
Density & $\rho$ & $2.59$ &3.06 &$\text{g}/\text{cm}^3$\\
Compressive strength & $\sigma_c$ & $120.54$&272.92 &$\text{MPa}   $\\
Tensile strength & $\sigma_t$ & $7.02$&16.61 &$ \text{MPa}   $\\
%static elastic modulus & $E_s$ & $50.00$& &$ \text{GPa}   $\\
Elastic modulus & $E$ & $49.75$&105.46 &$ \text{GPa}   $\\
Poisson's ratio & $\nu$ & 0.26 & 0.26  & -\\
Fracture toughness & $K_I$ & $0.95$& 2.61 &$\text{MPa}\cdot\text{m}^{0.5}$\\
Friction angle (Mohr) & $\phi$ &  $52.5$& 44 &$^\circ$\\
Cohesion & $c$ &  $22.5$& 25.00  &$ \text{MPa}   $\\
Basic friction angle &$\phi_b$ &30 & 31.2 & $^\circ$\\
\end{tabular}
\caption{Rock parameters of granite and basalt used in the direct shear tests.}
\label{table:MEX7_rockParam}
\end{center}
\end{table}

The elastic modulus and the compressive strength were determined using uni-axial compression tests. For the tensile strength a tension test was conducted and the fracture toughness is evaluated by a bending test. The cohesion was determined through a shear test using a saw-cut joint. Ultrasonic measurements were used to determine the Poisson's ratio.
%
The basic friction angles were determined based on \cite{Alejano20121023}. The inner friction angle of basalt is from \cite{Schultz19951} and of granite from \cite{Lanaro2005} and \cite{Ramana2019273}.

\subsubsection{Rock surface scanning}
\index{rock surface scanning}

Important measurements to characterise a rock surface are surface scans. In the laboratory of the TU Freiberg a white light scanner is used, see Fig. \ref{TUBAFScanner}. It is a non-contact method which uses monochromatic light. A fringe projection is done and surface scans of rock samples with a resolution of about $30-50 \unit{\mu m}$ can be calculated. Further details about the scanning device can be found in \cite{TUBAFScanningDevice}. 

\begin{figure}[!ht]
\centering
\includegraphics[width=1.4\textwidth, angle=90]{figures/geomint-wp3-12a}
\caption{Surface scanning. A rock surface is digitized using a surface scanning device. The result is a point cloud.}\label{TUBAFScanner}
\end{figure}
\clearpage
%---------------------------------------------------------------------------------------------------
%\FloatBarrier
%\clearpage
%===================================================================================================
\section{Thermo-Hydro-Mechanical Laboratory Tests}
%---------------------------------------------------------------------------------------------------
%% Fracture Toughness
\subsection{X-ray micro computed tomography}
\Authors{Matthias Ruf (UoS)}
X-ray micro computed tomography ({\textmu}XRCT) is a non-destructive imaging technique and provides the possibility to examine the inner structure of an object by creating a digital 3D image of the same. It is based on the mathematical combination, called reconstruction, of several radiograms which were acquired from different directions. Thereby, a radiogram represents the respective measured intensity values $I(L)$ of the X-rays at position $x = L$ after travelling through the object and can be related to the unattenuated X-rays intensity $I_0$ before the object at position $x = 0$ by the Beer-Lambert law. Under the assumption of a monochromatic X-ray beam this can be written as 
\begin{align*}
I(L) = I_0 \exp\left(-\int_0^L \mu(x) \mathrm{d}x \right)
\end{align*}
for an inhomogeneous material with the unkown material depending attenuation coefficients $\mu(x)$ which are determined during the reconstruction process, cf. \cite{Carmignato2018}.

In Figure~\ref{fig:CTsystem} the self-made open and modular {\textmu}XRCT system at the Institute of Applied Mechanics - Chair of Continuum Mechanics of the University of Stuttgart, see \textcolor{red}{\cite{Ruf2020}} is depicted. 
\begin{figure*}[ht]
\centering
\includegraphics[width=1.0\textwidth]{figures/exp_2_2_xray_main.png}
\caption{Overview of the {\textmu}XRCT-system. The light blue arrows show the possible moving directions of the motorized stages.}
\label{fig:CTsystem}
\end{figure*}
It consists of the three main components: The X-ray source (1), the X-ray detector (2) and the sample positioning system including the rotation stage in between (3). Like most industrial CT-systems the specimen is rotated during the scan and the remaining parts are fixed. The employed X-ray tube provides a maximum power up to \SI{80}{\watt} and at the same time a focal spot size down to \SI{3}{\micro\meter} for moderate power levels. The acceleration voltage of the tube can be adjusted in the range of \SI{30}{\kilo\volt} to \SI{180}{\kilo\volt} and the flux from \SI{10}{\micro\ampere} to \SI{1000}{\micro\ampere}. It can be chosen between two indirect conversion flat panel detectors with different characteristics and resolutions of $1944 \times 1536$ and $2940 \times 2304$ pixels. Both produce gray value images with a pixel depth of \SI{14}{bit} and each is separately mounted on high accurate, motorized XY stages. The latter offers the possibility to compensate for bad detector pixels by taking several images from slightly different detector positions and subsequently stitching of the same to improve the final image quality.

The geometric magnification $M$ is given by the relation of the source detector distance $SDD$ to the source object distance $SOD$, $M = SOD/SDD$, and can be adjusted in a wide range. Depending on the sample material and the smallest feature size of interest, the specimen's diameter can be up to \SI{100}{\milli\meter}. The maximum achievable spatial resolution of the system is about \SI{50}{lp/mm} at \SI{10}{\%} of the modulation transfer function (MTF) which means a smallest feature size of \SI{10}{\micro\meter} that can be resolved. The corresponding field of view for this case is \SI{5.88}{\milli\meter} in width and \SI{4.60}{\milli\meter} in height. Therefore, the smallest expedient sample diameter is about \SI{5}{\milli\meter}. The open and modular concept of the CT-system provides a broad range also for unconventional investigations.

In Figure~\ref{fig:exampleCarraraMarble} a CT scan of a cylindrical Carrara marble core with a diameter of \SI{5}{\milli\meter} for a high resolution scan is shown exemplarily.
\begin{figure*}[ht]
	\centering
    \begin{subfigure}[c]{0.49\textwidth}
    \centering
	\label{fig:exampleCarraraMarble3D}
	\includegraphics[width=0.9\textwidth]{figures/exp_2_2_scan_3d.png}
    \end{subfigure}
    \begin{subfigure}[c]{0.49\textwidth}
    \label{fig:exampleCarraraMarble2D}
	\includegraphics[width=0.9\textwidth]{figures/exp_2_2_scan_2d.png}
\end{subfigure}
\caption{CT scan of a Carrara marble core after thermal treatment.}
\label{fig:exampleCarraraMarble}
\end{figure*}
The visible micro-cracks along the grain boundaries were created by thermal treatment and are not present in the virgin state.  The geometric magnification was set to $M = 24.76$ which corresponds to the highest achievable spatial resolution and leads to a voxel size of \SI{2}{\micro\meter}. For additional details see \textcolor{red}{\cite{Ruf2020}}. Besides a qualitative assessment, the data sets offer the possibility of deriving several additional information by image processing. For instance, in geosciences the 3D pore characterization, the 3D grain analysis and the fracture analysis, cf. \cite{Cnudde2013}. 

%% -----------------Fracture toughness (CAU Kiel)
\subsection{Fracture Toughness of the Opalinus Claystone}
\label{sec:Fracture_Toughness_Exp}
\Authors{CAU Kiel}
In linear elastic fracture mechanics, a materials resistance to fracture propagation is known as fracture toughness. The unit of the fracture toughness is $Pa.\sqrt m$ and the fracture toughness is measured under coupled or individual three different fracture modes I, II and III. In a three-point bending test, the flexural strength ($\sigma_{flex}$), flexural Young's modulus ($E_{flex}$) and flexural strain ($\epsilon_{flex}$) parameters are measured. The fracture toughness test using the three-point bending test provides the Mode I fracture toughness ($K_{Ic}$) of the material (Figure \ref{fig:Amir_Fracture_Toughness_Theory}).

\begin{figure}[!ht]
\centering
\includegraphics[width=6cm,height=3cm]{figures/Amir_Fracture_Toughness_Theory.png}
\caption{The fracture toughness assessment using three-point bending test}
\label{fig:Amir_Fracture_Toughness_Theory}
\end{figure} 

The stress intensity factor ($K_I$) on the crack tip of predefined notch is obtained with (\ref{eq:Fracture_Toughness}) \cite{Bower2009},

\begin{multline}
\label{eq:Fracture_Toughness}
K_I=
\frac{4F_{flex}}{B_{flex}}
\sqrt{\frac{\pi}{W_{flex}}}
\left(1.6
\left(\frac{a_{flex}}{W_{flex}}\right)^\frac{1}{2}
-
2.6\left(\frac{a_{flex}}{W_{flex}}\right)^\frac{3}{2} 
\right.
\\ 
\left.
+12.3\left(\frac{a_{flex}}{W_{flex}}\right)^\frac{5}{2} -21.2\left(\frac{a_{flex}}{W_{flex}}\right)^\frac{7}{2}
+21.8\left(\frac{a_{flex}}{W_{flex}}\right)^\frac{9}{2}
\right)
\end{multline}

where, $F_{flex}$ is the flexural load, $a_{flex}$ is the length of the pre-defined notch, $B_{flex}$ and $W_{flex}$ are the thickness and height of the sample under the flexural test, respectively. During the test procedure, the load and crack mouth opening displacement (CMOD) values are measured and plotted. At the load in which the crack starts to propagate into the medium the fracture toughness $K_{Ic}$ is calculated. In order to perform the fracture toughness test, a loading cell with three rolling supports is required (Figure \ref{fig:Amir_Fracture_Toughness_Setup_a}). However, the in-situ condition can only be reached when a material is under controlled temperature and humidity conditions. Therefore, a climate chamber in a laboratory of CAU Kiel has been utilized to reach the desired temperature and humidity (Figure \ref{fig:Amir_Fracture_Toughness_Setup_b}). The temperature can be controlled between 20 to 80 $^{\circ}C$ and relative humidity varies from 0 up to 100. 

\begin{figure}[!ht]
\centering
\begin{subfigure}[c]{0.5\textwidth}
\centering
\includegraphics[width=6cm,height=5cm]{figures/Amir_Fracture_Toughness_Setup_a.png}
\subcaption{}
\label{fig:Amir_Fracture_Toughness_Setup_a}
\end{subfigure}
\hfill
\begin{subfigure}[c]{0.48\textwidth}
\centering
\includegraphics[width=4cm,height=5cm]{figures/Amir_Fracture_Toughness_Setup_b.png}
\subcaption{}
\label{fig:Amir_Fracture_Toughness_Setup_b}
\end{subfigure}
\caption{The required equipments for performing the three-point bending test (a) the loading cell and supports, (b) the climate chamber for controlling temperature and humidity}
\end{figure}

The claystone samples are prepared with the dimension of 140x30x30 $(mm)$ and the notch dimension of 2x10x30 $(mm)$v $(LxWxB)$  (Figure \ref{fig:Amir_Fracture_Toughness_Sample}). The span length ($S_{flex}$) is 120 $mm$, which is 4 times the size of its width and thickness. The embedded layering is perpendicular to the loading direction. The image processing technique is used to track the reference points, which are predefined prior to the test procedure (Figure \ref{fig:Amir_Fracture_Toughness_Image_a}). The distance between the points is measured using the optical microscopic image and is considered as an initial reading value (Figure \ref{fig:Amir_Fracture_Toughness_Image_b}). The method is able to detect the minimum displacement of 2 micrometers, which is possible by taking 4k video with 30fps. The load vs. CMOD response of the Opalinus claystone as well as the numerical simulation outcomes are given in \ref{sec:mex01}. The effects of anisotropy and embedded layering orientation of Opalinus claystones on fracture toughness values are not fully studied. 

\begin{figure}[!ht]
\centering
\includegraphics[width=8cm,height=4cm]{figures/Amir_Fracture_Toughness_Sample.png}
\caption{The prepared Opalinus claystone sample with the dimension of 140x30x30 $mm$}
\label{fig:Amir_Fracture_Toughness_Sample}
\end{figure} 

\begin{figure}[!ht]
\centering
\begin{subfigure}[c]{0.48\textwidth}
\centering
\includegraphics[width=6cm,height=5cm]{figures/Amir_Fracture_Toughness_Image_a.png}
\subcaption{}
\label{fig:Amir_Fracture_Toughness_Image_a}
\end{subfigure}
\hfill
\begin{subfigure}[c]{0.48\textwidth}
\centering
\includegraphics[width=6cm,height=5cm]{figures/Amir_Fracture_Toughness_Image_b.png}
\subcaption{}
\label{fig:Amir_Fracture_Toughness_Image_b}
\end{subfigure}
\caption{The application of image processing technique (a) the predefined reference points for measuring the CMOD, (b) the measured rough distance using optical microscope}
\end{figure}


%% ---------------------- Splitting Test (CAU Kiel)

\subsection{Brazilian Disk Test on Barrier Rocks}
\label{sec:Brazilian_Disk_Exp}
\Authors{CAU Kiel}
The tensile strength of a brittle or quasi-brittle material, such as rock, is one of the most important material properties, which in comparison to the compression strength, is much weaker. The measurement of direct tensile strength of brittle materials is difficult and time consuming. Therefore, finding the splitting tensile strength ($\sigma_{sp}$) is a fast alternative for computing the direct tensile strength. The Brazilian disk test is conducted to determine the splitting tensile strength (Figure \ref{fig:Amir_Splitting_Theory}). The splitting strength depends on loading rate, diameter and length of the specimen \ref{eq:Splitting_Strength}. The works of \cite{Perras2014} and \cite{Li2013} provides a full review and a correlation between different rock strength properties.

\begin{figure}[!ht]
\centering
\includegraphics[width=6cm,height=5cm]{figures/Amir_Splitting_Theory.png}
\caption{The Brazilian disk test on a cylindrical sample with diameter and length of $D_{cyl}$ and $L_{cyl}$, respectively}
\label{fig:Amir_Splitting_Theory}
\end{figure} 

\begin{align}
\label{eq:Splitting_Strength}
\begin{split}
\sigma_{sp}=\frac{2F_{sp}}{\pi L_{cyl}D_{cyl}}
\end{split}
\end{align}

In the CAU Kiel laboratory, the splitting strength under THM processes is measured. To do so, a controlled temperature and humidity climate chamber (Figure \ref{fig:Amir_Fracture_Toughness_Setup_b}) and a loading frame with a displacement transducer are required. Initially, the splitting test under room temperature condition with initial humidity of the sample is conducted. Afterwards, the temperature is raised to 50 and 80 $^{\circ}C$. The relative humidity can be controlled from 0 up to 100. Finally, the splitting strength, splitting Young’ modulus ($E_sp$) and load-displacement behavior are measured. The cylindrical claystone and saltstone samples are prepared in dimension of 100x200 $mm$ $(DxL)$ (Figure \ref{fig:Amir_Splitting_Sample}) and a strain rate of  0.1 \% is considered to insure a relatively fast loading rate. Figure \ref{fig:Amir_Splitting_Setup} shows the placement of the sample under loading frame before performing the test.

\begin{figure}[!ht]
\centering
\begin{subfigure}[c]{0.3\textwidth}
\centering
\includegraphics[width=4cm,height=5cm]{figures/Amir_Splitting_Sample.png}
\subcaption{}
\label{fig:Amir_Splitting_Sample}
\end{subfigure}
\hfill
\begin{subfigure}[c]{0.6\textwidth}
\centering
\includegraphics[width=6cm,height=5cm]{figures/Amir_Splitting_Setup.png}
\subcaption{}
\label{fig:Amir_Splitting_Setup}
\end{subfigure}
\caption{The sample preparation and test procedure (a) the claystone sample with a dimension of 100x200 $mm$, (b) the placement of saltstone inside of the climate chamber}
\end{figure}

Initially, the splitting strength of a saltstone samples under room temperature (20), 50 and 80 $^{\circ}C$ are measured. Due to a relatively homogeneous material property of the saltstone, the orientation of the sample will not affect the final outcomes. Figure \ref{fig:Amir_Splitting_Salt_20} and \ref{fig:Amir_Splitting_Salt_Result} depicts the failure in 20 $^{\circ}C$ and load vs. displacement for different loading temperatures, respectively. The observed failure pattern for all of the setups is identical. The measured mean splitting strength using \ref{eq:Splitting_Strength} for 20, 50 and 80 $^{\circ}C$ are 1.65, 1.59 and 1.43 $MPa$, respectively. As a result, the temperature has a slight influence on the splitting strength of saltstone, especially when the temperature is raised up to 50 $^{\circ}C$.

\begin{figure}[!ht]
\centering
\begin{subfigure}[c]{0.35\textwidth}
\centering
\includegraphics[width=4cm,height=5cm]{figures/Amir_Splitting_Salt_20.png}
\subcaption{}
\label{fig:Amir_Splitting_Salt_20}
\end{subfigure}
\hfill
\begin{subfigure}[c]{0.6\textwidth}
\centering
\includegraphics[width=6cm,height=5cm]{figures/Amir_Splitting_Salt_Result.png}
\subcaption{}
\label{fig:Amir_Splitting_Salt_Result}
\end{subfigure}
\caption{The splitting strength of a saltstone under different temperature conditions (a) the fracture path in saltstone in room temperature, (b) the load vs. displacement under different temperature conditions}
\end{figure}

The anisotropy of claystone and embedded layering orientation has a significant influence on material strength. To investigate the strength dependency on layering orientation, a series of tests, where the angle between the loading direction and layering orientation is 0 (parallel), 30, 45, and 90 (perpendicular) degree, is conducted. For 20 $^{\circ}C$, the failure pattern for 0, 45 and 90 degrees are provided in Figure \ref{fig:Amir_Splitting_Clay_0}, \ref{fig:Amir_Splitting_Clay_45} and \ref{fig:Amir_Splitting_Clay_90}. Figure \ref{fig:Amir_Splitting_Clay_20_Result} depicts the load vs. displacement behavior of claystone with different layering degrees. 

\begin{figure}[ht!]
\centering
\begin{subfigure}[c]{0.48\textwidth}
\centering
\includegraphics[width=5cm,height=5cm]{figures/Amir_Splitting_Clay_0.png}
\subcaption{}
\label{fig:Amir_Splitting_Clay_0}
\end{subfigure}
\hfill
\begin{subfigure}[c]{0.48\textwidth}
\centering
\includegraphics[width=5cm,height=5cm]{figures/Amir_Splitting_Clay_45.png}
\subcaption{}
\label{fig:Amir_Splitting_Clay_45}
\end{subfigure}
\hfill
\begin{subfigure}[c]{0.48\textwidth}
\centering
\includegraphics[width=5cm,height=5cm]{figures/Amir_Splitting_Clay_90.png}
\subcaption{}
\label{fig:Amir_Splitting_Clay_90}
\end{subfigure}
\caption{The fracture pattern under different layering orientation (a) 0 degrees, parallel, (b) 45 degrees, and (c) 90 degrees, perpendicular (20 $^{\circ}C$)}
\end{figure}

\begin{figure}[ht!]
\centering
\includegraphics[width=9cm,height=5cm]{figures/Amir_Splitting_Clay_20_Result.png}
\caption{The load vs. displacement under different layering orientation, 20 $^{\circ}C$}
\label{fig:Amir_Splitting_Clay_20_Result}
\end{figure} 

Table \ref{table:Amir_Splitting_Table1} presents the mean outcome of the experimental tests for different orientation angles and temperature. The results depicts that when the loading is perpendicular to the layering orientation, the splitting strength of the claystone is almost 5 times higher than when it is parallel to the layering orientation. The temperature effect on splitting strength is negligible. 

\begin{table}[!ht]
\centering
\begin{center}
\begin{tabular}{ | >{\centering\arraybackslash}X m{8em} | >{\centering\arraybackslash}X m{3em}| >{\centering\arraybackslash}X m{3em} | >{\centering\arraybackslash}X m{3em} | >{\centering\arraybackslash}X m{3em} | >{\centering\arraybackslash}X m{3em} | }
\hline
Test Results & 0 $^{\circ}$ & 30 $^{\circ}$ & 45 $^{\circ}$ & 60 $^{\circ}$ & 90 $^{\circ}$ \\
\hline
$\sigma_{sp}$ ($MPa$) 20$^{\circ}C$ & 0.47 & 0.68 & 1.13 & 1.45 & 1.92  \\ 
\hline
$\sigma_{sp}$ ($MPa$) 80$^{\circ}C$ & 0.52 & 0.64 & 1.02 & 1.25 & 1.86   \\
\hline
\end{tabular}
\end{center}
\caption{The splitting tensile strength of the Opalinus claystone with different temperature and layering orientations}
\label{table:Amir_Splitting_Table1}
\end{table}


%\clearpage
%---------------------------------------------------------------------------------------------------
%\subsection{Triaxial Tests on Cylindrical Samples (CAU Kiel)}
%---------------------------------------------------------------------------------------------------
\subsection{True Triaxial Test on the Cubic Opalinus Claystone Samples}
\label{sec:True_Triaxial_Exp}
\Authors{CAU Kiel}
The true triaxial apparatus, where the stresses are controlled along three axes, is used to investigate the three-dimensional strain-strength behavior of soil or rock geomaterial (Figure \ref{fig:Amir_TrueTriaxial_Apparatus}). The true triaxial device in the CAU Kiel laboratory is able to reach a mechanical loading of 600 $MPa$ as well as a thermal loading of up to 600 $^{\circ}C$. The cubic samples are prepared in the side dimension of 43 $mm$ and the edges are slightly curved in order to avoid the stress concentration and failure in the corners (Figure \ref{fig:Amir_TrueTriaxial_Sample}).

\begin{figure}[!ht]
\begin{subfigure}[c]{0.48\textwidth}
\includegraphics[width=1\textwidth]{figures/Amir_TrueTriaxial_Apparatus.png}
\subcaption{}
\label{fig:Amir_TrueTriaxial_Apparatus}
\end{subfigure}
\hfill
\begin{subfigure}[c]{0.48\textwidth}
\includegraphics[width=1\textwidth]{figures/Amir_TrueTriaxial_Sample.png}
\subcaption{}
\label{fig:Amir_TrueTriaxial_Sample}
\end{subfigure}
\caption{The (a) true triaxial apparatus in geomechanics laboratory of CAU Kiel, and (b) prepared cubic claystone sample with curved edges}
\end{figure}

The coupled thermo-mechanical loading conditions are considered in order to investigate the materials anisotropic stiffness along three axis, the elastic and plastic deformation under cyclic thermal and mechanical loadings, deviatoric stress field, and material failure. A thermal loading of up to $T_{iso}^{max}=150$ $^{\circ}C$ is considered, which is considered to be higher than the maximum temperature that the claystone samples are subjected to in nuclear waste disposal sites. The maximum isotropic mechanical loading is considered to be $\sigma_{iso}^{max}=100 MPa$. The considered boundary conditions are:


\begin{itemize}
  \item Sample MT-01: Mechanical Condition,5 loading cycles, $\sigma_{iso}^{max}=100 MPa$.
  \item Sample MT-02: Coupled Thermo-Mechanical Condition, 4 loading cycles, $T_{iso}^{max}=150$ $^{\circ}C$ and $\sigma_{iso}^{con}=12 MPa$.
  \item Sample MT-03: Coupled Thermo-Mechanical Condition, 4 loading cycles, $T_{iso}^{max}=150$ $^{\circ}C$, and $\sigma_{iso}^{max}=100 MPa$.
  \item Sample MT-04: Coupled Thermo-Mechanical Condition, 4 loading cycles, $\sigma_{dev}^{max}=60 MPa$, $\sigma_{conf}^{con}= 20 MPa$ and $T_{iso}^{con}=150$ $^{\circ}C$ .
\end{itemize}


With the installation of the ultrasonic sensors on the pistons, the apparatus is able to measure the ultrasonic $P$, $S90$ and $S0$ waves along the axis. The anisotropy factor, density ($\rho$), the dynamic Young’s modulus ($E_{Dyn}$), dynamic shear modulus($G_{Dyn}$), and dynamic Poisson’s ratio ($\nu_{Dyn}$) values can all be calculated according to the analytical relation which already exist in the literature \cite{Motraetal2018} as given in Eq. \ref{eq:YoungsModulus_Ultrasonic} and \ref{eq:PoissonsRatio_Ultrasonic}. In order to detect and analyse the ultrasonic signals in Opalinus Claystone, the minimum confining mechanical stresses of 12 $MPa$ is required. The test results of MT-01 (Figure \ref{fig:Amir_TrueTriaxial_MT_01_Result}) depicts the increment of the mean $E_{Dyn}$, $G_{Dyn}$ and $\nu_{Dyn}$ values under the isotopic confinement stresses up to 100 $MPa$, which is due to the layering orientation and structure of claystones and eventually improved contact qualities. It is also noted that the rate of the increment after each loading cycle (slope of the line) is decreased and due to the plastic deformations and improved contact qualities, the materials mechanical properties in initial loading condition of 12 $MPa$ are enhanced. 


\begin{figure}[ht!]
\centering
\begin{subfigure}[c]{0.48\textwidth}
\centering
\includegraphics[width=6cm,height=4cm]{figures/Amir_TrueTriaxial_MT_01_Result_E.png}
\subcaption{}
\end{subfigure}
\hfill
\begin{subfigure}[c]{0.48\textwidth}
\centering
\includegraphics[width=6cm,height=4cm]{figures/Amir_TrueTriaxial_MT_01_Result_G.png}
\subcaption{}
\end{subfigure}
\hfill
\begin{subfigure}[c]{0.48\textwidth}
\centering
\includegraphics[width=6cm,height=4cm]{figures/Amir_TrueTriaxial_MT_01_Result_Nu.png}
\subcaption{}
\end{subfigure}
\caption{The true triaxial results for the MT-01 sample under 5 loading cycles (a) $E_{Dyn}$ vs. Confinement Stress, (b) $G_{Dyn}$ vs. Confinement Stress, and (c) $\nu_{Dyn}$ vs. Confinement Stress}
\label{fig:Amir_TrueTriaxial_MT_01_Result}
\end{figure}


 Figure \ref{fig:Amir_TrueTriaxial_MT_02_Result} shows the deformation vs. time result of MT-02 in the 1st cycle. The thermal plastic deformation after 5 cycles of loading and unloading was negligible and can be neglected. The MT-02 sample is situated in a way that the loading frame in direction of Z is parallel to the layering orientations of claystone. The volumetric thermal expansion coefficient ($\alpha_V$) is calculated based on the measured volumetric strain ($\epsilon_V=\frac{\Delta V}{V}$) and temperature change ($\Delta T$). The effect of temperature on $\alpha_V$ is shown at Figure \ref{fig:Amir_TrueTriaxial_MT_02_Result_1a}. The rate of $\alpha_V$ is decreased after 80 $^{\circ}C$. The anisotropy in linear thermal expansion coefficient along Z, Y and X axis is shown in Figure \ref{fig:Amir_TrueTriaxial_MT_02_Result_1b}. The anisotropy in the measured linear expansion coefficient ($\alpha_L$) is substantially decreased after 80 $^{\circ}C$.

\begin{align}
\label{eq:ThermalExpansion}
\begin{split}
\alpha_V=\frac{\epsilon_V}{\Delta T}=\frac{\Delta V}{V\Delta T}
\end{split}
\end{align}

\begin{figure}[!ht]
\centering
\includegraphics[width=7cm,height=4cm]{figures/Amir_TrueTriaxial_MT_02_Result.png}
\caption{The deformation vs time for sample MT-02}
\label{fig:Amir_TrueTriaxial_MT_02_Result}
\end{figure} 

\begin{figure}[!ht]
\centering
\begin{subfigure}[c]{0.48\textwidth}
\centering
\includegraphics[width=6cm,height=4cm]{figures/Amir_TrueTriaxial_MT_02_Result_1a.png}
\subcaption{}
\label{fig:Amir_TrueTriaxial_MT_02_Result_1a}
\end{subfigure}
\hfill
\begin{subfigure}[c]{0.48\textwidth}
\centering
\includegraphics[width=6cm,height=4cm]{figures/Amir_TrueTriaxial_MT_02_Result_1b.png}
\subcaption{}
\label{fig:Amir_TrueTriaxial_MT_02_Result_1b}
\end{subfigure}
\caption{The effect of temperature on (a) $\alpha_V$, and (b) the anisotropy in the linear thermal expansion coefficient ($\alpha_L$)}
\end{figure}

The test results of MT-03 investigates the anisotropy of Opalinus claystone samples.Figure \ref{fig:Amir_TrueTriaxial_MT_03_Result} illustrates the $1^{th}$ cycle loading results of $E_{Dyn}$ and $\nu_{Dyn}$ values in X, Y and Z directions under the isotopic confinement stresses up to 100 $MPa$. The $G_{Dyn}$ has a similar behavior to the $E_{Dyn}$ values. The results indicate a weaker stiffness in Z direction, which is parallel to the layering orientation. The thermal loading upto 150 $^{\circ}C$ results in weaker $E_{Dyn}$ values during the heating process, which is the reason for the fluctuation of the results at the confinement stress of 100 $MPa$. The test results of MT-04 is similar to the MT-02 and the same material response is observed. 

\begin{figure}[ht!]
\centering
\begin{subfigure}[c]{0.48\textwidth}
\centering
\includegraphics[width=6cm,height=4cm]{figures/Amir_TrueTriaxial_MT_03_Result_E.png}
\subcaption{}
\end{subfigure}
\hfill
\begin{subfigure}[c]{0.48\textwidth}
\centering
\includegraphics[width=6cm,height=4cm]{figures/Amir_TrueTriaxial_MT_03_Result_Nu.png}
\subcaption{}
\end{subfigure}
\caption{The true triaxial results for the MT-03 sample in the $1^{th}$ cycle (a) $E_{Dyn}$ vs. Confinement Stress, and (b) $\nu_{Dyn}$ vs. Confinement Stress}
\label{fig:Amir_TrueTriaxial_MT_03_Result}
\end{figure}


%===================================================================================================
\section{Shrinkage and Swelling Laboratory Tests (WP1)}

\subsection{The Swelling Characteristic of the Opalinus Claystone}
\Authors{IfG}
\todo[inline]{[IfG](): The description of the experiment procedure}

%---------------------------------------------------------------------------------------------------
%\subsection{Oedometer Test (BGR)}
%--------------------------Shrinkage (CAU Kiel)
\subsection{The Wetting and Drying Paths of the Opalinus Claystone (CAU Kiel)}
\label{sec:Shrinkage_Swelling_Exp}
\Authors{CAU Kiel}

The shrinkage and swelling of claystone results in micro fracking and higher permeability values, which in nuclear waste disposal sites can lead to contamination of groundwater. Micro fracking also decreases the strength of the material subjected to THM  loading processes. Typically, the swelling pressure and heave of claystone is determined using Oedometer tests \cite{Peronetal2009}, where a constrained or unconstrained sample is subjected to the swelling process (Figure \ref{fig:Amir_Shrinkage_Swelling_Setup}). During the test procedure, the swelling pressure, as well as the heave magnitude, are recorded. It is also observed that the swelling pressure in sandy facies of claystone is lower than shaly facies. In contrast to the swelling tests, shrinkage tests are less common and are complicated to perform on rock materials. \cite{Minardietal2016} performed a shrinkage test on claystone with shaly and sandy facies using a desiccator and various salt solutions \ref{fig:Amir_Shrinkage_Minardi}. During the test procedure the axial strain values obtained from the strain gauges, as well as sample weight, are measured.


\begin{figure}[!ht]
\begin{subfigure}[c]{0.48\textwidth}
\includegraphics[width=1\textwidth]{figures/Amir_Shrinkage_Swelling_Setup.png}
\subcaption{}
\label{fig:Amir_Shrinkage_Swelling_Setup}
\end{subfigure}
\hfill
\begin{subfigure}[c]{0.48\textwidth}
\includegraphics[width=1\textwidth]{figures/Amir_Shrinkage_Minardi.png}
\subcaption{}
\label{fig:Amir_Shrinkage_Minardi}
\end{subfigure}
\caption{The swelling and shrinkage tests on Opalinus claystone (a) the Oedometer test setup for constrained swelling pressure in Opalinus clay samples \cite{Peronetal2009}, and (b) measuring the shrinkage and swelling paths using desiccator \cite{Minardietal2016}}
\end{figure}

Two prepared thin cylindrical sections of sandy Opalinus claystone (Mont-Terri) with a dimension of 100x10 $mm (DxH)$ are used to determine the anisotropy in shrinkage and swelling behavior of claystone. Due to the mineral structure of claystone and its layering orientation, the anisotropy factor has a significant influence on the direction of shrinkage and swelling as well as the micro fracking formation. The initial sample is used to determine the axial strains in parallel and perpendicular directions to the layering orientations \ref{fig:Amir_Shrinkage_Sensors}. The strain gauge strips (HBM, $LY 10 (mm)$/120 $\Omega$) are glued and attached to the surface of the sample. The second sample is used to measure the weight change in the sample under different salt solutions. The saturated salt solutions are located inside the desiccator and will result in osmotic suction and eventually in the wetting and drying of the samples (Figure \ref{fig:Amir_Shrinkage_Full_Setup}). 

\
\begin{figure}[!ht]
\begin{subfigure}[c]{0.48\textwidth}
\includegraphics[width=1\textwidth]{figures/Amir_Shrinkage_Sensors.png}
\subcaption{}
\label{fig:Amir_Shrinkage_Sensors}
\end{subfigure}
\hfill
\begin{subfigure}[c]{0.48\textwidth}
\includegraphics[width=1\textwidth]{figures/Amir_Shrinkage_Full_Setup.png}
\subcaption{}
\label{fig:Amir_Shrinkage_Full_Setup}
\end{subfigure}
\caption{The test preparation using two sandy Opalinus claystone samples (a) the placement of the strain gauge strips on claystone surface, and (b) the desiccator setup}
\end{figure}

The total suction($\psi_{total}$) value can be measured using the Kelvin’s relation, which is derived from ideal gas law:

\begin{equation}
\label{eq:Total_Suction}
\psi_{total} = \frac{R_{gas}T}{V_{mol}} \ln(\frac{P_{vapor}^{curvature}}{P_{vapor}^{curvature=0}})
\end{equation}

where, $(\frac{P_{vapor}^{curvature}}{P_{vapor}^{curvature=0}})$ is the relative humidity, $P_{vapor}$ is the vapor pressure, $R_{gas}$ is the universal gas constant, $T$ is the temperature and $V_{mol}$ is the molecular volume of water. The considered salt solution and its induced suction and relative humidity values in the constant room temperature of 20 $^{\circ}C$ are listed in Table \ref{table:Amir_Shrinkage_SaltSolutions}. The equilibrium inside the desiccator is reached when the strain gauges value or water content of the samples are constant in two consecutive readings. 


\begin{table}[h!]
\centering
\begin{center}
\begin{tabular}{ |>{\centering\arraybackslash}X m{7em}|>{\centering\arraybackslash}X m{10 em}|>{\centering\arraybackslash}X m{7em}|} 
\hline
Salt Solution & Relative Humidity (\%) & Suction ($MPa$) \\
\hline
$K_2SO_4$ & 97.6 & 3.2 \\
\hline
$KNO_3$ & 94.6 & 7.5 \\
\hline
$KCl$ & 85.1 & 21.8 \\
\hline
$NaCl$ & 75.5 & 38\\
\hline
$Mg(NO_3)_2$ & 54 & 84 \\
\hline
$MgCl_2$ & 33.1 & 149.5 \\
\hline
$LiCl$ & 12 & 286.7\\
\hline
$LiBr$ & 6.6 & 367.5\\
\hline
\end{tabular}
\end{center}
\caption{The saturated salt solutions relative humidity and suction at 20 $^{\circ}C$}
\label{table:Amir_Shrinkage_SaltSolutions}
\end{table}

The results obtained from the drying and wetting paths of the Opalinus claystone are given in \ref{sec:mex06}, where a numerical simulation and comparison to the experimental data are provided. The test procedure is time consuming and after measuring the data for more than 120 days, the applied suction and water content percentages are calculated and plotted. 

%---------------------------------------------------------------------------------------------------
\subsection{In-situ Condition Desiccation Process (Stuttgart or BGR)}
\todo[inline]{[UoS/BGR](): More description please}
Topology, respectively morphology investigations of fractures induced in clay stone throughout drying processes require a number of experimental sequences ranging from the drying process itself to characterization of the induced fractures by X-Ray Computed Tomography. The required samples are prepared and characterized by the University of Kiel with a dimension of radius $r = 5 \, \text{mm}$ and height $h = 50 \, \text{mm}$ to guarantee the realization of scans in Regions of Interests (RoIs) down to a characteristic edge length of $6 \, \text{mm}$ resulting in a resolution of 2 micrometers per voxel.

Two different experimental set-ups are planned; nevertheless the first proposed experimental implementation is preferred over the second.

\subsection{Drying Wet Samples - From Wet to Dry}
\todo[inline]{[UoS](): More description please}
Saturated samples are prepared by the University of Kiel and submerged in a shrink tube in order to minimize the change of the desired state. Once the experiments are performed on the sample the shrink tube is removed and the probe is installed in an uni-axial testing device. The sample is installed in a heat chamber and loaded by an axial force of $f_{ax} = 500 \, \text{N}$/, resulting in an axial stresses of $ \sigma_{ax} = 0.25 \text{MPa}$ (potentially up to $f_{ax} = 5 \, \text{kN}$/$ \sigma_{ax} = 2.5 \text{MPa}$) while axial deformations are measured under controlled temperature conditions. By measuring the sample's weight at characteristic states (twice - beginning and end) the volumetric deformations can be determined and related to the change of saturation. The sample characterization is then completed by XRCT scans to determine the features of the induced fractures. Finally the stiffness degradation is determined by ultrasound experiments measuŕing the P-wave run times at 2, respectively 6 MHz.

\subsection{Wetting dry Samples - From Dry to Wet}
\todo[inline]{[UoS](): More description please}

Since it is extremely time consuming to increase the saturation of the sample this second setting is only an alternative to the introduced set-up in case the first experiment fails. The set-up is comparable to the proposed experiment but performed in a heat-humidity chamber at rel. humidity between 10 - 90 \% and temperatures between $5-90^\circ C$. The applied axial Force would reduce to $f_{ax} = 50 \, \text{N}$.

%===================================================================================================
\section{Pressure Driven Percolation Laboratory Tests (WP2)}
%---------------------------------------------------------------------------------------------------
\subsection{Pressure Driven Percolation}
\Authors{IfG}
The determination of the permeability, which quantifies the flow behavior of fluids in the pore space of a rock, is based on the 
Darcy equation:

\begin{equation}
q = \frac{kA}{L\eta}\Delta p
\end{equation}

with:

$k$ : permeability (m$^2$) \\
$A$ : cross sectional area (m$^2$) \\
$l$ : length of sample (m) \\
$\eta$ : dynamic viscosity (Pa$\cdot$ s) \\
$q$ : flow rate (m$^3$/s) \\
$\Delta p$ : pressure difference (Pa)

Thereafter, the flow rate of a fluid through a sample at a given pressure differential is measured by the viscosity of 
the flowing medium, the geometric factor of the sample, and the permeability (with the dimension of an area). The 
permeability is given as SI-unit in m2 or, traditionally, in D (Darcy) (1 D corresponds to about 10$^{-12}$ m$^2$).

The equation above is only valid for incompressible fluids, because only then is the flow rate $q$ constant over the flow path. 
With sufficient accuracy, this applies to low compressible fluids.
When a gas flows through the pore space of a solid, however, an expansion of the gas takes place along the flow path, so that the 
flow rate is not constant here. In this case, instead of the flow rate $q$ the mean flow rate $q_m$ is set, for which, for small 
pressures with sufficient accuracy according to the law of Boyle-Mariotte 

\begin{equation}
q_m p_m = q_0 p_0
\end{equation}

with:

$q_m$ : mean flow rate \\
$p_m$ : mean pressure \\
$q_0$ : measured flow rate at $p_0$ \\
$p_0$ : pressure at flow rate measurement 

If, for the mean pressure $p_m$, the arithmetic mean of the pressures $p_1$ on the high pressure side and $p_2$ on the low 
pressure side of the porous solid is used, taking into account that $\Delta p = p_1 - p_2$, the modified Darcy equation for 
gas flows is as follows:

\begin{equation}
k = \frac{2p_0q_0\eta L}{A(p_1^2-p_2^2}
\end{equation}

In summary, the determination of the permeability with caustic or gas with knowledge of the viscosity requires in each 
case an exact measurement of the flow rate after setting (quasi) stationary flow conditions and the pressure gradient.

To carry out permeability tests, the IfG Leipzig uses a servohydraulic testing machine with a pressure cell up to pc-max = 1000 bar, 
which is otherwise used for strength tests with $F_{max}$ = 2500 kN (manufacturer: Schenk / Trebel) (Fig. \ref{fig:ifglabph4}). The tests 
routinely set hydrostatic pressure conditions ($\sigma_1 = \sigma_2 = \sigma_3$), but it is also possible to implement 
deviatoric stresses or defined deformations. The axial load or deformation and the jacket pressure are each controlled 
independently via a servo-hydraulic system. 

The desired jacket pressure is generated by a pressure intensifier. From the axial deformation and the measured change in 
volume of the lateral pressure chamber (piston displacement of the pressure booster), the volume change of the test specimen, 
referred to here as dilatancy, can be determined at constant jacket pressure.

In hydrostatic loads sintered metal plates are used, which allow over the entire cross-sectional area of the sample a fluid 
pressurization. The pressure measurement of the measuring fluid is carried out by pressure transducers from Hottinger 
(accuracy class 0.2), whereby depending on the measuring range a 20 bar or 200 bar encoder is used.

\begin{figure}[!ht]
\centering
\includegraphics[width=1\textwidth]{./figures/ifg-lab-photo4.png}
\caption{Triaxial IfG pressure cell for flow tests with brine or gas.}
\label{fig:ifglabph4}
\end{figure}

The measuring principle for permeability determination under stationary conditions is based on the measurement of the 
flow rate $q$, here under atmospheric conditions ($p_0$), in the axial sample direction at a predetermined pressure gradient $\Delta p = p_1 - p_2$ ($p_1$ = inlet 
pressure, $p_2$ = outlet pressure).

The measuring arrangement used in the triaxial cell has the following advantages over the frequently used Hassler cells, in 
which a cylindrical sample is firmly clamped between punches and is only subjected to radial pressure.

\begin{itemize}
\item Determination of sample deformation during hydrostatic and deviatoric pressurization
\item Use of variable sample geometries (stamp sets between 60 and 110 mm diameter are available, height up to 2 x diameter)
\item The height of the gas injection pressure is limited only by the available gas cylinder pressure (routinely max: 200 bar).
\item For higher gas pressures (up to 1000 bar), a Maximator pneumatic pressure booster is used, but this is only used for special measurements.
\end{itemize}

For the determination of gas permeability two methods are available:

1st: Injection of nitrogen at a defined injection rate within a range of min. 0.1 ml / min to max. 500 ml / min under 
measurement of the injection pressure when stationary conditions are reached with leakage of the fluid against the atmosphere.

2nd: Injecting nitrogen under a defined pre-pressure and measuring the flow rate at the exit side of the sample against the atmosphere.

\begin{figure}[!ht]
\centering
\includegraphics[width=1\textwidth]{./figures/ifg-perme-flowrate.png}
\caption{Variation diagram of permeability vs. flow rate for a sample with l = 220 mm and d = 110 mm at different injection pressures.}
\label{fig:ifgpermeflow}
\end{figure}

The lower limit of the measuring range depends essentially on the pre-pressure or the minimum measurable flow rate, as shown 
schematically in Fig. \ref{fig:ifgpermeflow}.

For gas permeability measurements, EL-FLOW mass flow controllers or flow meters from Bronkhorst are used with the following specifications:

\begin{itemize}
\item Mass flow controller Type: F-230M: Measuring range: (0) ... 10 ... 500 ml / min N2
\item Form: 200 bar / outlet pressure: 194 bar / temperature: 20$^\circ$C
\item Measuring accuracy: ± 1\% of final value, typ. better 0.5\%
\item Mass flow controller Type: F-230M: Measuring range: (0) ... 0.4 ... 20 ml / min
\end{itemize}



%---------------------------------------------------------------------------------------------------

\subsection {The Fluid Driven Percolation Tests on Cubic Opalinus Claystone samples from Mont-Terri}
\label{sec:Percolation_Claystone_Exp}
\Authors{CAU Kiel}
The investigation of a fluid transport in claystone due to its anisotropic behavior and its role as a rock barrier in nuclear waste repositories has a great importance. Performing hydraulic fracking under pressurized fluid or storing pressurized fluids leads to the fracking of rock barrier and fluid transport through the hydraulic apertures and cavities. This can lead to pressure and volume drop in the reservoir, decrease the output and efficiency of the designed system and the contamination of ground water. In the geomechanics laboratory of CAU Kiel, the true triaxial apparatus with the maximum mechanical pressure of 600 $MPa$ and thermal loading up to 600 $^{\circ}C$ is used to conduct the fluid driven percolation in claystone samples from Mont-Terri \ref{fig:Amir_TrueTriaxial_Apparatus}. The syringe pump, with the maximum pressure of 517 $bar$, is used to pressurize the oil fluid. The cubic samples with the side dimension of 43 $mm$ and center hole length and diameter of 20 and 8 $mm$ are prepared \ref{fig:Amir_Percolation_Adapter}, respectively, and attached to the pump pipes, where the sealing is done with O-rings and epoxy glue \ref{fig:Amir_Percolation_Setup}. 


\begin{figure}[!ht]
\begin{subfigure}[c]{0.48\textwidth}
\includegraphics[width=6cm,height=4cm]{figures/Amir_Percolation_Adapter.png}
\subcaption{}
\label{fig:Amir_Percolation_Adapter}
\end{subfigure}
\hfill
\begin{subfigure}[c]{0.48\textwidth}
\includegraphics[width=6cm,height=4cm]{figures/Amir_Percolation_Setup.png}
\subcaption{}
\label{fig:Amir_Percolation_Setup}
\end{subfigure}
\caption{The fluid driven percolation test preparation (a) the prepared cubic claystone sample and the adapter¸ and (b) the sample placement inside the true triaxial apparatus}
\end{figure}


Two different stress configurations, as well as fluid injection directions parallel or perpendicular to the layering orientation, are considered to investigate the fracking pattern as well as flow pathways. The fracking tests are carried out under a constant fluid pressure and the peak fluid pressure, where the flow rate increases and any sudden drops in fluid pressure is recorded. In the initial test, a sample with a fluid injection direction perpendicular to the layering orientation of the Opalinus sample is considered \ref{fig:Amir_Percolation_Orientation1}. The initial stress configuration is 12, 14 and 16 $MPa$ in three different loading directions \ref{fig:Amir_Percolation_Stress_1}. In order to prevent damaging the sample prior to the hydraulic fracking test, the isotropic stresses of 8 $MPa$ is applied from all pistons and is then gradually increased to the planned stress configuration. The oil pressure is increased gradually up until the point where the borehole pressure drops and an increase in the flow rate can be seen. The test is then aborted immediately in order to avoid causing any damage to the true triaxial apparatus. In the second test setup, a sample with a fluid injection direction parallel to the layering orientation of the Opalinus sample is considered \ref{fig:Amir_Percolation_Orientation2}. The initial stress configuration is 16, 10 and 8 $MPa$ in three different loading directions \ref{fig:Amir_Percolation_Stress_2}.


\begin{figure}[!ht]
\begin{subfigure}[c]{0.48\textwidth}
\includegraphics[width=4cm,height=4cm]{figures/Amir_Percolation_Orientation1.png}
\subcaption{}
\label{fig:Amir_Percolation_Orientation1}
\end{subfigure}
\hfill
\begin{subfigure}[c]{0.48\textwidth}
\includegraphics[width=5cm,height=4cm]{figures/Amir_Percolation_Stress_1.png}
\subcaption{}
\label{fig:Amir_Percolation_Stress_1}
\end{subfigure}
\caption{The boundary conditions for the $1^{th}$ stress configuration (a) the orientation of the embedded layers, and (b) the initial stress configuration}
\end{figure}

\begin{figure}[!ht]
\begin{subfigure}[c]{0.48\textwidth}
\includegraphics[width=4cm,height=4cm]{figures/Amir_Percolation_Orientation2.png}
\subcaption{}
\label{fig:Amir_Percolation_Orientation2}
\end{subfigure}
\hfill
\begin{subfigure}[c]{0.48\textwidth}
\includegraphics[width=5cm,height=4cm]{figures/Amir_Percolation_Stress_2.png}
\subcaption{}
\label{fig:Amir_Percolation_Stress_2}
\end{subfigure}
\caption{The boundary conditions for the $2^{nd}$ stress configuration  (a) the orientation of the embedded layers, and (b) the initial stress configuration}
\end{figure}

The results of the percolation tests on the Opalinus claystone samples are given in \ref{sec:mex02}, where a numerical simulation and comparison to the experimental data are provided and the effect of stress distribution, anisotropy and layering orientation on frack paths and fracking pressure are discussed. 


%---------------------------------------------------------------------------------------------------
\subsection{Fluid Driven Percolation and Healing in Saltstone}
\Authors{IfG}
\todo[inline]{[IfG](): Description (Amir:check the comments on this section)}

 removing the section as we discussed in our meeting, however, add a small paragraph to the chapter 2.4.1 regarding the healing process
%---------------------------------------------------------------------------------------------------

\subsection{Effect of Compressibility in Pressure Driven Percolation}
\Authors{IfG}
\todo[inline]{[IfG](): Description (Amir:check the comments on this section)}

removing the section as we discussed in our meeting, however, add a small paragraph to the chapter 2.4.1 regarding the effect of compressiblity

extra comment: change the title of 2.4.1 to be more specific (to show the difference between 2.4.1 and 2.4.2). I suggest: Pressure Driven Percolation Tests on Cylindrical Samples

%===================================================================================================
\section{Stress Redistribution Laboratory Tests (WP3)}
%---------------------------------------------------------------------------------------------------
\subsection{Direct Shear Test}
%---------------------------------------------------------------------------------------------------
\Authors{TU Freiberg}
To conduct direct shear tests special equipment is necessary. The direct shear testing device at the rock mechanical laboratory of the TU Freiberg (see Fig. \ref{fig:ExpCNLShearMachine}) is specially developed to ensure the wanted functionality.\\

\begin{figure}[!ht]
\begin{center}
\includegraphics[width=0.5\textwidth]{./figures/ExpShearMachine.jpg}
\end{center}
\caption{The shear testing device at the rock mechanical laboratory at TU Freiberg. (From: \cite{Konietzky2012})}
\label{fig:ExpCNLShearMachine}
\end{figure}

Some key features can be found in Tab. \ref{table:ExpCNLDeviceTechnicalData}. Additionally it is possible to superimpose dynamic forces. In the tests for the GeomInt project this functionality wasn't used.\\

\begin{table}[!ht]
\begin{center}
\begin{tabular}{l r r}
feature & value & unit\\
\hline
Max. normal force & 1000 & kN\\
Max. shear displacement & 50 &mm\\
Min. shear velocity & 1e-7 & mm/s\\
Max. shear velocity & 70 & mm/s\\
Max. sample size (rectangular) & 200$\times$400 & mm\\
Max. fluid pressure & 10 & MPa\\
\end{tabular}
\caption{Technical data of the shear testing device. (From: \cite{Konietzky2012})}
\label{table:ExpCNLDeviceTechnicalData}
\end{center}
\end{table}

The sample preparation includes the cutting of the rock block in the cuboid shape. This sample is split into two parts and the rock joints are arranged in a matching position. It is arranged in the shear box, Fig. \ref{fig:ExpCNLSampleInShearBox}.\\


\begin{figure}[!ht]
\begin{center}
\includegraphics[width=0.5\textwidth]{./figures/ExpCNLSampleInShearBox.jpg}
\end{center}
\caption{Sample in shear box. (From: \cite{Nguyen2014})}
\label{fig:ExpCNLSampleInShearBox}
\end{figure}

The sample is grouted in the shear box to avoid any unwanted movements of it, Fig. \ref{fig:ExpCNLGroutedSample}.\\

\begin{figure}[!ht]
\begin{center}
\includegraphics[width=0.5\textwidth]{./figures/ExpCNLGroutedSample.jpg}
\end{center}
\caption{Grouted sample before direct shear test. (From: \cite{Nguyen2014})}
\label{fig:ExpCNLGroutedSample}
\end{figure}

The finally equipped shear box is connected to the measuring units, the LVDTs (Linear Variable Differential Transformer), Fig. \ref{fig:ExpCNLLVDT}. The accuracy of this length measurements is in the order of $\unit[]{\mu m}$.

\begin{figure}[!ht]
\begin{center}
\includegraphics[width=0.5\textwidth]{./figures/ExpCNLLVDT.jpg}
\end{center}
\caption{Measuring equipment: LVDT. (From: \cite{Nguyen2014})}
\label{fig:ExpCNLLVDT}
\end{figure}

The set-up of the experiment is the same for CNL and CNS test. In the CNS test the stiffness which adds an extra load is calculated. This means if a normal displacement of the sample is measured the normal stress will be adapted according to the defined stiffness.


%\subsection{CNS Direct Shear Test (TU Freiberg)}
%---------------------------------------------------------------------------------------------------
\subsection{Cyclic Loading Pressure Diffusion}
\Authors{University of Stuttgart}
In order to study characteristic, time-dependent states of fractures under cycling loading conditions a sample is prepared with a single fracture and borehole before it is installed in a triaxial cell as shown in figure \ref{fig:exp_cyclic_pressure_triax}. The dimension of the sample are chosen to be $r = 30 \, \text{mm}$ and height $h = 70 \, \text{mm}$. 
\begin{figure}[!ht]
\begin{center}
\includegraphics[width=0.5\textwidth]{./figures/exp_cyclic_pressure_triax.png}
\end{center}
\caption{Set-Up of triaxial cell.}
\label{fig:exp_cyclic_pressure_triax}
\end{figure}
The experiment is performed in three steps. After applying a confining pressure $p_{c}$ the initial state is approached by deformation control while the acting normal forces are measured in a first step. Once a desired normal force is reached the deformation state is held constant and a fluid pressure of $p_{fix}$ is applied. Finally, the fracture is stimulated by a harmonic fluid pressure with a frequency of $0.1 \, \text{Hz}$ and varying amplitudes $p_A$ between $0.5-3 \, \text{MPa}$. Throughout the experiment flow and pressure are measured at the fluid induction point to study the relationship of pressure and flow under non-constant fracture permeabilities triggered by deformations.