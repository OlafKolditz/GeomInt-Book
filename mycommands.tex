%Table of Contents (TOC)
\setcounter{tocdepth}{2} 
\sethlcolor{magenta}
\newcommand{\hljm}[1]{{\sethlcolor{olive}\hl{#1}}}
\newcommand{\hlmn}[1]{{\sethlcolor{lime}\hl{#1}}}
\newcommand{\hltf}[1]{{\sethlcolor{cyan}\hl{#1}}}
\newcommand{\hlwpi}[1]{{\sethlcolor{olive}\hl{#1}}}
\newcommand{\hlwpii}[1]{{\sethlcolor{lime}\hl{#1}}}
\newcommand{\hlwpiii}[1]{{\sethlcolor{cyan}\hl{#1}}}
\newcommand{\hlok}[1]{{\sethlcolor{magenta}\hl{#1}}}
\newcommand{\hltn}[1]{{\sethlcolor{green}\hl{#1}}}
\newcommand{\hlbgr}[1]{{\sethlcolor{olive}\hl{#1}}}
\newcommand{\hlcau}[1]{{\sethlcolor{yellow}\hl{#1}}}
\newcommand{\hlifg}[1]{{\sethlcolor{lime}\hl{#1}}}
\newcommand{\hltuf}[1]{{\sethlcolor{cyan}\hl{#1}}}
\newcommand{\hlufz}[1]{{\sethlcolor{magenta}\hl{#1}}}
\newcommand{\hluos}[1]{{\sethlcolor{lightgray}\hl{#1}}}
\newcommand{\hlman}[1]{{\sethlcolor{pink}\hl{#1}}}
%
\captionsetup[table]{skip=5pt}
%
\newcommand{\Authors}[1]{{\textit{#1}\\[2mm]}}
%
\newcommand{\todol}[1]{{\todo[color=black]{\textcolor{black}{#1}}}}
\newcommand{\todod}[1]{{\todo[color=green]{\textcolor{green}{#1}}}}
\newcommand{\todold}[1]{{\todo[color=yellow]{\textcolor{black}{#1}}}}

%%%Math
\newcommand{\beq}{\begin{equation}}
\newcommand{\eeq}{\end{equation}}
\newcommand{\nn}{\nonumber}
\newcommand{\mwith}{\quad \text{with} \quad}
\newcommand{\mand}{\quad \text{and} \quad}
\newcommand{\mom}{\quad \text{on} \quad}
\newcommand{\mdiv}{\,\text{div}\,}
\newcommand{\mDiv}{\,\text{Div}\,}
\newcommand{\grad}{\,\text{grad}\,}
\newcommand{\Grad}{\,\text{Grad}\,}
\newcommand{\Cel}{\,$^\circ$C}
\newcommand{\dcdot}{\mbf{\,:\,}}
\newcommand{\tr}{\text{tr}\,}
\newcommand{\mathd}{\text{d}}
\newcommand{\mathD}{\text{D}}
\newcommand{\mdiag}{\,\text{diag\,}}
\newcommand{\fourtens}[1]{\mbox{${\mbox{\boldmath$\mathcal{#1}$\unboldmath}}$}}
\newcommand{\mbf}[1]{{\mathbf{#1}}}
\newcommand{\mbfs}{\boldsymbol}
\newcommand{\dev}{\stackrel{def}{=}}
\newcommand{\tbf}{\textbf}
\newcommand{\tit}{\textit}
\newcommand{\mrm}{\text}
%\newcommand{\citep}[1]{(\cite{#1})}
%\newcommand{\citet}{\cite}
\newcommand{\tf}{$\rightarrow\ $}
\newcommand{\ds}{\displaystyle}
\newcommand{\mtd}[2]{\frac{\mathd_{#1}{#2}}{\mathd t}} %material time derivative
\newcommand{\ptd}[1]{\frac{\partial {#1}}{\partial t}} %partial time derivative
\newcommand{\pd}[2]{\frac{\partial {#1}}{\partial {#2}}} %partial derivative
\newcommand{\vint}[1]{\int \limits_\Omega {#1} \mathd \Omega} %volume integral
\newcommand{\sint}[2]{\int \limits_{\partial \Omega_{#1}} {#2} \mathd \Gamma}
\newcommand{\uexp}[1]{$^{\text{{#1}}}$}%superscript
\newcommand{\uind}[1]{$_{\text{{#1}}}$}%subscript
\newcommand{\mvec}[1]{\mathsfbfit{#1}}%Vektoren
\newcommand{\mmat}[1]{\mathsfbfit{#1}}%Matrizen
\newcommand{\iter}[3]{\ {}^{#3}{#1}^{#2}}%quantity, time, iteration
\newcommand{\drop}[1]{\red \cancelto{0}{\black #1} \black}
\newcommand{\centerpic}[3]{\hspace{#1\textwidth}\resizebox{#2\textwidth}{!}{\includegraphics{#3}}}
\newcommand{\squote}[2]{\begin{quote}{\huge{``}}{#1}{\huge{''}}\end{quote}\hfill Aus: {#2}}
\newcommand{\mpar}[1]{\marginpar{\flushleft \color{red} \tiny #1}}
%math
\newcommand{\md}[1]{\mtd{\mrm{S}}{#1}}
%OK ToDo
\newcommand{\uline}[1]{\underline{#1}}
\newcommand{\uuline}[1]{\underline{\underline{#1}}}
\newcommand{\p}{\partial}
\newcommand{\myparagraph}[1]{\paragraph{#1}\mbox{}\\} %linebreak after paragraphs
%
%\usepackage{float}  %fix the position of figures
\newcommand{\jump}[1]{[\![#1]\!]}
\newcommand{\ATone}{\texttt{AT$_1$}}
\newcommand{\ATtwo}{\texttt{AT$_2$}}
\newcommand{\norm}[1]{\left\lVert#1\right\rVert}
%% Shortcuts Patrick
% Derivatives
\newcommand{\pt}{\partial}
\newcommand{\pderiv}[2]{\frac{\pt #1}{\pt #2}}
\newcommand{\ppderiv}[2]{\frac{\pt^2 #1}{\pt #2^2}}
\newcommand{\mtderivfill}[1]{\left(\frac{\text{D} #1}{\text{D} \, t \hfill}\right)}
\newcommand{\mtderiv}[1]{\frac{\text{D} #1}{\text{D} \, t}}
%% SPH Commands
%==============================================================================
% Abk"urzungen
%==============================================================================
\newcommand{\CalA}      {{\mathcal A}}
\newcommand{\CalB}      {{\mathcal B}}
\newcommand{\CalC}      {{\mathcal C}}
\newcommand{\CalD}      {{\mathcal D}}
\newcommand{\CalE}      {{\mathcal E}}
\newcommand{\CalF}      {{\mathcal F}}
\newcommand{\CalG}      {{\mathcal G}}
\newcommand{\CalH}      {{\mathcal H}}
\newcommand{\CalI}      {{\mathcal I}}
\newcommand{\CalJ}      {{\mathcal J}}
\newcommand{\CalK}      {{\mathcal K}}
\newcommand{\CalL}      {{\mathcal L}}
\newcommand{\CalM}      {{\mathcal M}}
\newcommand{\CalN}      {{\mathcal N}}
\newcommand{\CalO}      {{\mathcal O}}
\newcommand{\CalP}      {{\mathcal P}}
\newcommand{\CalQ}      {{\mathcal Q}}
\newcommand{\CalR}      {{\mathcal R}}
\newcommand{\CalS}      {{\mathcal S}}
\newcommand{\CalT}      {{\mathcal T}}
\newcommand{\CalU}      {{\mathcal U}}
\newcommand{\CalV}      {{\mathcal V}}
\newcommand{\CalW}      {{\mathcal W}}
\newcommand{\CalX}      {{\mathcal X}}
\newcommand{\CalY}      {{\mathcal Y}}
\newcommand{\CalZ}      {{\mathcal Z}}
\def\res                {\mbox{Res}     }
\def\Res                {\mbox{Res}     }
\newcommand{\RB}        {Rand\-be\-ding\-ung}
\newcommand{\DRB}       {\Name{Dirichlet}-\RB}
\newcommand{\NRB}       {\Name{Neumann}-\RB}
%==============================================================================
% Namen normal und fett
%==============================================================================
% Bold Small Caps als Macro \bsc
\newsavebox{\bscbox}
\newlength{\bscwidth}
\newlength{\bscshift}
\setlength{\bscshift}{0.1pt}
\newcommand{\bsc}[1]{%
  \savebox{\bscbox}{\sc #1}%
  \settowidth{\bscwidth}{\usebox{\bscbox}}%
  \hspace*{-\bscshift}\usebox{\bscbox}\hspace*{-\bscwidth}%
  \hspace*{-\bscshift}\usebox{\bscbox}\hspace*{-\bscwidth}%
  \hspace*{4\bscshift}\usebox{\bscbox}\hspace*{-\bscwidth}%
  \hspace*{-\bscshift}\usebox{\bscbox}\hspace*{-\bscwidth}%
  \hspace*{-\bscshift}\raisebox{2\bscshift}{\usebox{\bscbox}}%
}
% Bold Slanted als Macro \bsl
\newsavebox{\bslbox}
\newlength{\bslwidth}
\newlength{\bslshift}
\setlength{\bslshift}{0.1pt}
\newcommand{\bsl}[1]{%
  \savebox{\bslbox}{\sl #1}%
  \settowidth{\bslwidth}{\usebox{\bslbox}}%
  \hspace*{-\bslshift}\usebox{\bslbox}\hspace*{-\bslwidth}%
  \hspace*{-\bslshift}\usebox{\bslbox}\hspace*{-\bslwidth}%
  \hspace*{4\bslshift}\usebox{\bslbox}\hspace*{-\bslwidth}%
  \hspace*{-\bslshift}\usebox{\bslbox}\hspace*{-\bslwidth}%
  \hspace*{-\bslshift}\raisebox{2\bslshift}{\usebox{\bslbox}}%
}
%\newcommand{\Name}[1]  {{\sc #1}}
%\newcommand{\NAME}[1]  {{\bsc #1}}
%\newcommand{\Name}[1]   {{{#1}}\/}
\newcommand{\NAME}[1]   {{\bsl{#1}$\,$}\/}
\newcommand{\Fett}[1]   {{\bfseries #1}}
\newcommand{\FETT}[2]   {{\bfseries #1}}
%==============================================================================
% Umgebungen wie Def, Satz oder Beweis
%==============================================================================
%\newtheorem{Def}{Definition}[chapter]
%\newtheorem{Satz}[Def]{Satz}
%\newenvironment{Bewei\medskips}{{\bf Beweis:}}{\Qed\smallskip}
%\newenvironment{Beispiel}{{\bf Beispiel:}}{}
%\newenvironment{Bemerkung}{\smallskip{\bf Bemerkung:}}{\hfill$\Box$\smallskip}
\newsavebox{\BildText}
\newenvironment{Bild}[1]{\savebox{\BildText}{#1}\begin{figure}[htb]}%
                        {\centerline{\usebox{\BildText}}\end{figure}}
%\renewcommand{\thetable}   {\thechapter.\arabic{table}}
%\renewcommand{\thefigure}  {\thechapter.\arabic{figure}}
%\renewcommand{\theequation}{\thechapter.\arabic{equation}}
\newcommand{\Kasten}[3]%
{\begin{equation}
  \fbox{
    \begin{minipage}{5.5in}
      \centering
      #2
      \rule{5.4in}{0.5pt}
      {\bf #3}
    \end{minipage}
  }
  \label{box:#1}
\end{equation}}
%--- Referenzen
\newcommand{\Ref}[1]%
{%
  \typeout{Referenz #1 in Seite \thepage}%
  \ifthenelse{\equal{\pageref{#1}}{\thepage}}%
             {\ref{#1}}%
             {\ref{#1} auf Seite \pageref{#1}}%
}
\newcommand{\RefBraces}[1]%
{%
  \typeout{Referenz #1 in Seite \thepage}%
  \ifthenelse{\equal{\pageref{#1}}{\thepage}}%
             {(\ref{#1})}%
             {(\ref{#1}) auf Seite \pageref{#1}}%
}
\newcommand{\refpart}[1]{\Ref{part:#1}}         % Referenz eines Teils
\newcommand{\refchap}[1]{\Ref{chap:#1}}         % Referenz eines Kapitels
\newcommand{\refsect}[1]{\Ref{sect:#1}}         % Referenz eines Abschnitts
\newcommand{\refssec}[1]{\Ref{ssec:#1}}         % Referenz eines Unterabschn.
\newcommand{\refssub}[1]{\Ref{ssub:#1}}         % Referenz eines Unterunterab.
\newcommand{\refdef}[1] {\Ref{def:#1}}          % Referenz einer Definition
\newcommand{\reftab}[1] {\Ref{tab:#1}}          % Referenz einer Tabelle
\newcommand{\reffig}[1] {\Ref{fig:#1}}          % Referenz eines Bildes
\newcommand{\refeqn}[1] {\RefBraces{eqn:#1}}    % Referenz einer Gleichung
\newcommand{\refbox}[1] {\RefBraces{box:#1}}    % Referenz eines Kastens
\newcommand{\Cite}[2]{\Name{#1} \cite{#2}}
\newcommand{\CITE}[3]{\Name{#1} \cite[#3]{#2}}
%==============================================================================
% Mathematische Operatoren/Funktionen
%==============================================================================
\newcommand{\Abs   }[1]{    \left|  #1  \right| }
\newcommand{\LRN   }[1]{    \left(  #1  \right) }
\newcommand{\LRE   }[1]{    \left[  #1  \right] }
\newcommand{\LRG   }[1]{    \left\{ #1  \right\}}
\newcommand{\NORM  }[1]{    \left\| #1  \right\|}
\newcommand{\Norm  }[1]{         \| #1        \|}
% Ausserdem gibt es standard-maessig in LaTeX:
% \log \lg \ln \lim \limsup \liminf
% \sin \arcsin \sinh \cos \arccos \cosh
% \tan \arctan \tanh \cot \coth \sec \csc
% \max \min \sup \inf
% \arg \ker \dim \hom \deg
% \det \exp \gcd
\def\adj{\mathop{\mathrm{adj}}\nolimits}
\def\cof{\mathop{\mathrm{cof}}\nolimits}
\def\dev{\mathop{\mathrm{dev}}\nolimits}
\def\vol{\mathop{\mathrm{vol}}\nolimits}
\def\diag{\mathop{\mathrm{diag}}\nolimits}
\def\div{\mathop{\mathrm{div}}\nolimits}
\def\Div{\mathop{\mathrm{Div}}\nolimits}
\def\grad{\mathop{\mathrm{grad}}\nolimits}
\def\gradT{\mathop{{\mathrm{grad}}^T}\nolimits}
\def\Grad{\mathop{\mathrm{Grad}}\nolimits}
\def\Grada{\mathop{\mathrm{Grad}}_\alpha\nolimits}
\def\GradS{\mathop{{\mathrm{Grad}}_S}\nolimits}
\def\GradST{\mathop{{\mathrm{Grad}}_S^T}\nolimits}
\def\range{\mathop{\mathrm{range}}\nolimits}
\def\rank{\mathop{\mathrm{rank}}\nolimits}
\def\sym{\mathop{\mathrm{sym}}\nolimits}
\def\tr{\mathop{\mathrm{tr}}\nolimits}
\def\lin{\mathop{\mathrm{lin}}\nolimits}
\def\const{\mathop{\mathrm{const}}\nolimits}
\newcommand{\leqapprox}{\raisebox{-.5ex}{\large$\stackrel{<}{\Ss\sim}$}}
%==============================================================================
% Umgebungen
%==============================================================================
\newcommand{\bit}       {\begin{itemize}}
\newcommand{\eit}       {\end{itemize}}
\newcommand{\bde}       {\begin{description}}
\newcommand{\ede}       {\end{description}}
\newcommand{\bec}       {\begin{center}}
\newcommand{\eec}       {\end{center}}
\newcommand{\benum}     {\begin{enumerate}}
\newcommand{\eenum}     {\end{enumerate}}
%==============================================================================
% Kurzkommandos
%==============================================================================
\newcommand{\Ol}[1]{\overline{#1}}              % Ueberstreichung
\newcommand{\Ul}[1]{\underline{#1}}             % Unterstreichung
\newcommand{\Nn}{\nonumber\\}                   % Formeln ohne Nummern
\newcommand{\Ss}{\scriptstyle}                  % kleiner Formelsatz
\newcommand{\Ds}{\displaystyle}                 % grosser Formelsatz
\newcommand{\Bm}{\boldmath}                     % fette Formeln
\newcommand{\Bs}[1]{\boldsymbol{#1}}            % fette Symbole
\newcommand{\Sum}{\sum\limits}                  % Summe    mit voller Hoehe
\newcommand{\Int}{\int\limits}                  % Integral mit voller Hoehe
\newcommand{\Qed}{\hfill \mbox{$\Box$}}         % qed bei Beweisen
\newcommand{\Auf}{\longrightarrow}              % Abbildung 'auf'
\newcommand{\Und}{\ \land \ }                   % Logisches 'Und'
\newcommand{\Oder}{\ \lor  \ }                  % Logisches 'Oder'
\newcommand{\OInt}{\oint\limits}                % Integral mit unten/oben
\newcommand{\Gegen}{\rightarrow}                % Grenzwert 'gegen'
\newcommand{\Folgt}{\Longrightarrow}            % Daraus    'folgt'
\newcommand{\Gleich}{\Longleftrightarrow}       % "aquivalent zu
%==============================================================================
% Schriftendefinitionen ...
%==============================================================================
%\DeclareMathAlphabet\mathsfm{OT1}{cmss}{m}{n}
%\DeclareMathAlphabet\mathsfb{OT1}{cmss}{bx}{n}                                                        
%==============================================================================
% Meta-Kommandos zur einfachen Umstellung
%==============================================================================
\newcommand{\SetP}[1]{{\cal #1}}                % Symbole Punktmengen
\newcommand{\TenN}[1]{{\rm #1}}                 % Symbole Tensorrechnung normal
\newcommand{\TenB}[1]{{\rm\bf #1}}              % Symbole Tensorrechnung fett
\newcommand{\TenZ}[1]{{\eurb #1}}               % Symbole Tensorrechnung Zweip.
\newcommand{\TenG}[1]{\Bs{#1}}                  % Symbole Tens. griech.  fett
\newcommand{\TenF}[1]{\stackrel{4}{{\eurb #1}}} % Symbole Tensorrechnung 4-st.
\newcommand{\Set }[1]{\mathbb{#1}}              % Symbole Mengenzeichen
\newcommand{\MatN}[1]{{\mathsfm{#1}}}           % Symbole Matrizenrechnung      normal
%\newcommand{\MatB}[1]{{\mathsfb{#1}}}           % Symbole      dto              fett                   
%-----
%\newcommand{\MatN}[1]{{#1}}                     % Symbole Matrixrechnung normal
%\newcommand{\Bs}[1]{\boldsymbol{#1}}           % fette Symbole
\newcommand{\MatB}[1]{\Bs{#1}}                  % Symbole Matrixrechnung fett
%==============================================================================
% Mengen-Zeichen
%==============================================================================
\newcommand{\Q}{\Set{Q}}
\newcommand{\E}{\Set{E}}
\newcommand{\R}{\Set{R}}
\newcommand{\Z}{\Set{Z}}
%==============================================================================
% Tensorrechnung
%==============================================================================
\def\vec  #1{\mbox{\boldmath $#1$}{}}
\newcommand{\Phase}     {\varphi^\alpha}
\newcommand{\Ja}        {J_\alpha}
\newcommand{\Kin}       {\TenG{\chi}}
\newcommand{\Kina}      {\TenG{\chi}_\alpha}
%\newcommand{\Kin}      {{\Phi}}
%\newcommand{\Kina}     {{\Phi}_\alpha}
\newcommand{\PB}        {\SetP{B}}
\newcommand{\PBa}       {\SetP{B}^\alpha}
\newcommand{\PM}        {\SetP{M}}
\newcommand{\PS}        {\SetP{S}}
\newcommand{\PU}        {\SetP{U}}
\newcommand{\TXB}       {T_\TX\PB}
\newcommand{\TXBs}      {T^*_\TX\PB}
\newcommand{\TXaB}      {T_{\TX_\alpha}\!\PB}
\newcommand{\TXaBs}     {T^*_{\TX_\alpha}\!\PB}
\newcommand{\TxS}       {T_\Tx\PS}
\newcommand{\TxSs}      {T^*_\Tx\PS}
%\newcommand{\TB}        {T\PB}
\newcommand{\TBs}       {T^*\PB}
%\newcommand{\TS}        {T\PS}
\newcommand{\TSs}       {T^*\PS}
\newcommand{\ZF}        {\TenZ{F}}
\newcommand{\ZFa}       {\TenZ{F}_\alpha}
\newcommand{\ZR}        {\TenZ{R}}
\newcommand{\ZRa}       {\TenZ{R}_\alpha}
\newcommand{\dm}        {{\rm d}m}
\newcommand{\dt}        {{\rm d}t}
\newcommand{\dtau}        {{\rm d}\tau}
\newcommand{\dv}        {{\rm d}v}
\newcommand{\dV}        {{\rm d}V}
\newcommand{\dw}        {{\rm d}w}
\newcommand{\dW}        {{\rm d}W}
\newcommand{\dx}        {{\rm d}x}
\newcommand{\da}        {{\rm d}a}
\newcommand{\dalpha}    {{\rm d}\alpha}
\newcommand{\dA}        {{\rm d}A}
\newcommand{\dom}        {{\rm d}\omega}
\newcommand{\tennull}   {\TenB{0}}
\newcommand{\vecnull}   {\mbox{\small\bf{0}}}
%\newcommand{\I}        {\TenN{I}}
\newcommand{\IID}       {\TenN{II}_D}
\newcommand{\IIID}      {\TenN{III}_D}
\newcommand{\Tdx}       {\TenB{dx}}
\newcommand{\TdX}       {\TenB{dX}}
\newcommand{\TdXa}      {\TenB{dX}_\alpha}
\newcommand{\Tdv}       {\TenB{dv}}
\newcommand{\TEINS}     {\TenB{1}}
\newcommand{\TA}        {\TenB{A}}
\newcommand{\TAa}       {\TenB{A}_\alpha}
\newcommand{\TB}        {\TenB{B}}
\newcommand{\TC}        {\TenB{C}}
\newcommand{\TD}        {\TenB{D}}
\newcommand{\TE}        {\TenB{E}}
\newcommand{\TF}        {\TenB{F}}
\newcommand{\TG}        {\TenB{G}}
\newcommand{\TI}        {\TenB{I}}
\newcommand{\TL}        {\TenB{L}}
\newcommand{\TK}        {\TenB{K}}
\newcommand{\TM}        {\TenB{M}}
\newcommand{\TP}        {\TenB{P}}
\newcommand{\TQ}        {\TenB{Q}}
\newcommand{\TR}        {\TenB{R}}
\newcommand{\TS}        {\TenB{S}}
\newcommand{\TT}        {\TenB{T}}
\newcommand{\TW}        {\TenB{W}}
\newcommand{\TTa}       {\TenB{T}^\alpha}
\newcommand{\TU}        {\TenB{U}}
\newcommand{\TV}        {\TenB{V}}
\newcommand{\TVa}       {\TenB{V}_\alpha}
\newcommand{\TX}        {\TenB{X}}
\newcommand{\TXa}       {\TenB{X}_\alpha}
\newcommand{\TY}        {\TenB{Y}}
\newcommand{\Ta}        {\TenB{a}}
\newcommand{\Taa}       {\TenB{a}_\alpha}
\newcommand{\Tb}        {\TenB{b}}
\newcommand{\Tc}        {\TenB{c}}
\newcommand{\Tda}       {\TenB{d}_\alpha}
\newcommand{\Te}        {\TenB{e}}
\newcommand{\Th}        {\TenB{h}}
\newcommand{\Tg}        {\TenB{g}}
\newcommand{\Tj}        {\TenB{j}}
\newcommand{\Tk}        {\TenB{k}}
\newcommand{\Tn}        {\TenB{n}}
\newcommand{\Tm}        {\TenB{m}}
\newcommand{\Tp}        {\TenB{p}}
\newcommand{\Tq}        {\TenB{q}}
\newcommand{\Tr}        {\TenB{r}}
\newcommand{\Ts}        {\TenB{s}}
\newcommand{\Tt}        {\TenB{t}}
\newcommand{\Tf}        {\TenB{f}}
\newcommand{\Tu}        {\TenB{u}}
\newcommand{\Tv}        {\TenB{v}}
\newcommand{\Tva}       {\TenB{v}_\alpha}
\newcommand{\Tw}        {\TenB{w}}
\newcommand{\Tx}        {\TenB{x}}
\newcommand{\Txdot}     {\TenB{\dot{x}}}
\newcommand{\Ty}        {\TenB{y}}
\newcommand{\Tz}        {\TenB{z}}
\newcommand{\TPsi}      {\vec{\Psi}}
\newcommand{\TPhi}      {\vec{\Phi}}
\newcommand{\Tchi}      {\TenG{\chi}}
\newcommand{\Tlambda}   {\TenG{\lambda}}
\newcommand{\Tvarphi}   {\TenG{\vec{\varphi}}}
\newcommand{\Tphi}      {\TenG{\phi}}
\newcommand{\Tzeta}     {\TenG{\vec{\zeta}}}
\newcommand{\Teps}      {\TenG{\varepsilon}}
\newcommand{\Tgamma}      {\TenG{\gamma}}
\newcommand{\Tnu}       {\TenG{\nu}}
\newcommand{\Tsig}      {\TenG{\sigma}}
\newcommand{\Ttau}      {\TenG{\tau}}
\newcommand{\TXi}       {\TenG{\Xi}}
\newcommand{\Txi}       {\TenG{\xi}}
\newcommand{\TCF}       {\TenF{C}}
\newcommand{\Tkappa}    {\TenG{\vec{\kappa}}}
\newcommand{\TTheta}    {\TenG{\vec{\Theta}}}
\newcommand{\Tomega}    {\TenG{\vec{\omega}}}
\newcommand{\TOmega}    {\TenG{\vec{\Omega}}}
% dimensionslose Gr"o"sen mit "Uberstreichung
\newcommand{\Ddiv}      {\Ol{\div}}
\newcommand{\Dgrad}     {\Ol{\TenN{gra}}\TenN{d}\,}
\newcommand{\Dn}        {\bar{n}}
\newcommand{\Dp}        {\bar{p}}
\newcommand{\DTT}       {\bar{\TT}}
\newcommand{\DTX}       {\bar{\TX}}
\newcommand{\DTb}       {\bar{\Tb}}
\newcommand{\DTu}       {\bar{\Tu}}
\newcommand{\DTv}       {\bar{\Tv}}
\newcommand{\DTw}       {\bar{\Tw}}
\newcommand{\DTx}       {\bar{\Tx}}
\newcommand{\EuNum}     {{\mathrm{Eu}}}
\newcommand{\FrNum}     {{\mathrm{Fr}}}
\newcommand{\ReNum}     {{\mathrm{Re}}}
% spezielle Tensoren fuer TPM
\newcommand{\TepsS  }{\Teps_{\!S}^{\phantom{1}}}
\newcommand{\TepsSp }{\Teps_{\!Sp}^{\phantom{1}}}
\newcommand{\TepsSe }{\Teps_{\!Se}^{\phantom{1}}}
\newcommand{\TepsSpp}{(\Teps_{\!Sp}^{\phantom{1}})'_S}
\newcommand{\TepsSP }{\Teps_{\!Sp}}
\newcommand{\TTFE   }{\TT^F_{\!E}}
\newcommand{\TTSE   }{\TT^S_{\!E}}
\newcommand{\TTaE   }{\TT^\alpha_{\!E}}
% materielle Zeitableitungen im Rahmen der TPM
\newcommand{\MTx}        {\stackrel{'}{\Tx}}
%==============================================================================
% Matrizenrechnung
%==============================================================================
\newcommand{\Mzero}     {\MatB{0}}
\newcommand{\Mchi}      {\MatB{\chi}}
\newcommand{\Meps}      {\MatB{\varepsilon}}
\newcommand{\Meta}      {\MatB{\eta}}
\newcommand{\Mpsi}      {\MatB{\psi}}
\newcommand{\MLambda}   {\MatB{\Lambda}}
\newcommand{\MA}        {\MatB{A}}
\newcommand{\MB}        {\MatB{B}}
\newcommand{\ME}        {\MatB{E}}
\newcommand{\MF}        {\MatB{F}}
\newcommand{\MG}        {\MatB{G}}
\newcommand{\MJ}        {\MatB{J}}
\newcommand{\MM}        {\MatB{M}}
%%%
\newcommand{\ML}        {\MatB{L}}
\newcommand{\MP}        {\MatB{P}}
\newcommand{\MRes}      {\MatB{Res}}
%%%
\newcommand{\MN}        {\MatB{N}}
\newcommand{\MRi}       {\MatB{R}_i}
\newcommand{\MQ}        {\MatB{Q}}
\newcommand{\MQi}       {\MatB{Q}_{\!i}}
\newcommand{\MQj}       {\MatB{Q}_{\!j}}
\newcommand{\MQni}      {\MatB{Q}_{\!ni}}
\newcommand{\MQnj}      {\MatB{Q}_{\!nj}}
\newcommand{\MQns}      {\MatB{Q}_{\!ns}}
\newcommand{\MSi}       {\MatB{S}_i}
\newcommand{\MT}        {\MatB{T}}
\newcommand{\MU}        {\MatB{U}}
\newcommand{\MX}        {\MatB{X}}
\newcommand{\MXi}       {\MatB{X}_{\!i}}
\newcommand{\MY}        {\MatB{Y}}
\newcommand{\MYi}       {\MatB{Y}_{\!\!i}}
\newcommand{\MYj}       {\MatB{Y}_{\!\!j}}
\newcommand{\MYni}      {\MatB{Y}_{\!\!ni}}
\newcommand{\MYnj}      {\MatB{Y}_{\!\!nj}}
\newcommand{\MYns}      {\MatB{Y}_{\!\!ns}}
\newcommand{\MZi}       {\MatB{Z}_i}
\newcommand{\MZni}      {\MatB{Z}_{\!ni}}
\newcommand{\Me}        {\MatB{e}}
\newcommand{\Mf}        {\MatB{f}}
\newcommand{\Mg}        {\MatB{g}}
\newcommand{\Mh}        {\MatB{h}}
\newcommand{\Mj}        {\MatB{j}}
\newcommand{\Mk}        {\MatB{k}}
\newcommand{\MK}        {\MatB{K}}
\newcommand{\Mp}        {\MatB{p}}
\newcommand{\Mq}        {\MatB{q}}
\newcommand{\Mr}        {\MatB{r}}
\newcommand{\Mri}       {\MatB{r}_i}
\newcommand{\Mqp}       {\MatB{q}'}
\newcommand{\Mqn}       {\MatB{q}_n}
\newcommand{\Msi}       {\MatB{s}_i}
\newcommand{\Mt}        {\MatB{t}}
\newcommand{\Mu}        {\MatB{u}}
\newcommand{\Mup}       {\MatB{u}'}
\newcommand{\Mun}       {\MatB{u}_n}
\newcommand{\Mv}        {\MatB{v}}
\newcommand{\Mw}        {\MatB{w}}
\newcommand{\Mx}        {\MatB{x}}
\newcommand{\My}        {\MatB{y}}
\newcommand{\Myp}       {\MatB{y}'}
\newcommand{\Myn}       {\MatB{y}_{\!n}}
\newcommand{\Mz}        {\MatB{z}}
\newcommand{\Mzi}       {\MatB{z}_i}
%==============================================================================
% Matrizenrechnung - Symbold normal
%==============================================================================
\newcommand{\MNA}       {\MatN{A}}
\newcommand{\MNB}       {\MatN{B}}
\newcommand{\MNC}       {\MatN{C}}
\newcommand{\MND}       {\MatN{D}}
\newcommand{\MNE}       {\MatN{E}}
\newcommand{\MNF}       {\MatN{F}}
\newcommand{\MNG}       {\MatN{G}}
\newcommand{\MNH}       {\MatN{H}}
\newcommand{\MNI}       {\MatN{I}}
%==============================================================================
% Eulerfrakturen - Skalare und Index
%==============================================================================
\newcommand{\Ga}{\mathfrak{a}}
\newcommand{\Gb}{\mathfrak{b}}
\newcommand{\Gc}{\mathfrak{c}}
\newcommand{\Gd}{\mathfrak{d}}
\newcommand{\Ge}{\mathfrak{e}}
\newcommand{\Gf}{\mathfrak{f}}
\newcommand{\Gg}{\mathfrak{g}}
\newcommand{\Gh}{\mathfrak{h}}
\newcommand{\Gi}{\mathfrak{i}}
\newcommand{\Gj}{\mathfrak{j}}
\newcommand{\Gk}{\mathfrak{k}}
\newcommand{\Gl}{\mathfrak{l}}
\newcommand{\Gm}{\mathfrak{m}}
\newcommand{\Gn}{\mathfrak{n}}
\newcommand{\Go}{\mathfrak{o}}
\newcommand{\Gp}{\mathfrak{p}}
\newcommand{\Gq}{\mathfrak{q}}
\newcommand{\Gr}{\mathfrak{r}}
\newcommand{\Gs}{\mathfrak{s}}
\newcommand{\Gt}{\mathfrak{t}}
\newcommand{\Gu}{\mathfrak{u}}
\newcommand{\Gv}{\mathfrak{v}}
\newcommand{\Gw}{\mathfrak{w}}
\newcommand{\Gx}{\mathfrak{x}}
\newcommand{\Gy}{\mathfrak{y}}
\newcommand{\Gz}{\mathfrak{z}}
%
\newcommand{\GA}{\mathfrak{A}}
\newcommand{\GB}{\mathfrak{B}}
\newcommand{\GC}{\mathfrak{C}}
\newcommand{\GD}{\mathfrak{D}}
\newcommand{\GE}{\mathfrak{E}}
\newcommand{\GF}{\mathfrak{F}}
\newcommand{\GG}{\mathfrak{G}}
\newcommand{\GH}{\mathfrak{H}}
\newcommand{\GI}{\mathfrak{I}}
\newcommand{\GJ}{\mathfrak{J}}
\newcommand{\GK}{\mathfrak{K}}
\newcommand{\GL}{\mathfrak{L}}
\newcommand{\GM}{\mathfrak{M}}
\newcommand{\GN}{\mathfrak{N}}
\newcommand{\GO}{\mathfrak{O}}
\newcommand{\GP}{\mathfrak{P}}
\newcommand{\GQ}{\mathfrak{Q}}
\newcommand{\GR}{\mathfrak{R}}
\newcommand{\GS}{\mathfrak{S}}
\newcommand{\GT}{\mathfrak{T}}
\newcommand{\GU}{\mathfrak{U}}
\newcommand{\GV}{\mathfrak{V}}
\newcommand{\GW}{\mathfrak{W}}
\newcommand{\GX}{\mathfrak{X}}
\newcommand{\GY}{\mathfrak{Y}}
\newcommand{\GZ}{\mathfrak{Z}}
%==============================================================================
% Eulerfrakturen - FETT: Vektoren und Tensoren
%==============================================================================
\newcommand{\TGa}{\boldsymbol{\mathfrak{a}}}
\newcommand{\TGb}{\boldsymbol{\mathfrak{b}}}
\newcommand{\TGc}{\boldsymbol{\mathfrak{c}}}
\newcommand{\TGd}{\boldsymbol{\mathfrak{d}}}
\newcommand{\TGe}{\boldsymbol{\mathfrak{e}}}
\newcommand{\TGf}{\boldsymbol{\mathfrak{f}}}
\newcommand{\TGg}{\boldsymbol{\mathfrak{g}}}
\newcommand{\TGh}{\boldsymbol{\mathfrak{h}}}
\newcommand{\TGi}{\boldsymbol{\mathfrak{i}}}
\newcommand{\TGj}{\boldsymbol{\mathfrak{j}}}
\newcommand{\TGk}{\boldsymbol{\mathfrak{k}}}
\newcommand{\TGl}{\boldsymbol{\mathfrak{l}}}
\newcommand{\TGm}{\boldsymbol{\mathfrak{m}}}
\newcommand{\TGn}{\boldsymbol{\mathfrak{n}}}
\newcommand{\TGo}{\boldsymbol{\mathfrak{o}}}
\newcommand{\TGp}{\boldsymbol{\mathfrak{p}}}
\newcommand{\TGq}{\boldsymbol{\mathfrak{q}}}
\newcommand{\TGr}{\boldsymbol{\mathfrak{r}}}
\newcommand{\TGs}{\boldsymbol{\mathfrak{s}}}
\newcommand{\TGt}{\boldsymbol{\mathfrak{t}}}
\newcommand{\TGu}{\boldsymbol{\mathfrak{u}}}
\newcommand{\TGv}{\boldsymbol{\mathfrak{v}}}
\newcommand{\TGw}{\boldsymbol{\mathfrak{w}}}
\newcommand{\TGx}{\boldsymbol{\mathfrak{x}}}
\newcommand{\TGy}{\boldsymbol{\mathfrak{y}}}
\newcommand{\TGz}{\boldsymbol{\mathfrak{z}}}
%	     	 	     
\newcommand{\TGA}{\boldsymbol{\mathfrak{A}}}
\newcommand{\TGB}{\boldsymbol{\mathfrak{B}}}
\newcommand{\TGC}{\boldsymbol{\mathfrak{C}}}
\newcommand{\TGD}{\boldsymbol{\mathfrak{D}}}
\newcommand{\TGE}{\boldsymbol{\mathfrak{E}}}
\newcommand{\TGF}{\boldsymbol{\mathfrak{F}}}
\newcommand{\TGG}{\boldsymbol{\mathfrak{G}}}
\newcommand{\TGH}{\boldsymbol{\mathfrak{H}}}
\newcommand{\TGI}{\boldsymbol{\mathfrak{I}}}
\newcommand{\TGJ}{\boldsymbol{\mathfrak{J}}}
\newcommand{\TGK}{\boldsymbol{\mathfrak{K}}}
\newcommand{\TGL}{\boldsymbol{\mathfrak{L}}}
\newcommand{\TGM}{\boldsymbol{\mathfrak{M}}}
\newcommand{\TGN}{\boldsymbol{\mathfrak{N}}}
\newcommand{\TGO}{\boldsymbol{\mathfrak{O}}}
\newcommand{\TGP}{\boldsymbol{\mathfrak{P}}}
\newcommand{\TGQ}{\boldsymbol{\mathfrak{Q}}}
\newcommand{\TGR}{\boldsymbol{\mathfrak{R}}}
\newcommand{\TGS}{\boldsymbol{\mathfrak{S}}}
\newcommand{\TGT}{\boldsymbol{\mathfrak{T}}}
\newcommand{\TGU}{\boldsymbol{\mathfrak{U}}}
\newcommand{\TGV}{\boldsymbol{\mathfrak{V}}}
\newcommand{\TGW}{\boldsymbol{\mathfrak{W}}}
\newcommand{\TGX}{\boldsymbol{\mathfrak{X}}}
\newcommand{\TGY}{\boldsymbol{\mathfrak{Y}}}
\newcommand{\TGZ}{\boldsymbol{\mathfrak{Z}}}
%==============================================================================
% Index in Schreibschrift
%==============================================================================
\newcommand{\Ca}{\mbox{\calligra{\scriptsize a}}\/}
\newcommand{\Cb}{\mbox{\calligra{\scriptsize b}}\/}
\newcommand{\Cc}{\mbox{\calligra{\scriptsize c}}\/}
\newcommand{\Cd}{\mbox{\calligra{\scriptsize d}}\/}
\newcommand{\Ce}{\mbox{\calligra{\scriptsize e}}\/}
\newcommand{\Cf}{\mbox{\calligra{\scriptsize f}}\/}
\newcommand{\Cg}{\mbox{\calligra{\scriptsize g}}\/}
\newcommand{\Ch}{\mbox{\calligra{\scriptsize h}}\/}
\newcommand{\Ci}{\mbox{\calligra{\scriptsize i}}\/}
\newcommand{\Cj}{\mbox{\calligra{\scriptsize j}}\/}
\newcommand{\Ck}{\mbox{\calligra{\scriptsize k}}\/}
\newcommand{\Cl}{\mbox{\calligra{\scriptsize l}}\/}
\newcommand{\Cm}{\mbox{\calligra{\scriptsize m}}\/}
\newcommand{\Cn}{\mbox{\calligra{\scriptsize n}}\/}
\newcommand{\Co}{\mbox{\calligra{\scriptsize o}}\/}
\newcommand{\Cp}{\mbox{\calligra{\scriptsize p}}\/}
\newcommand{\Cq}{\mbox{\calligra{\scriptsize q}}\/}
\newcommand{\Cr}{\mbox{\calligra{\scriptsize r}}\/}
\newcommand{\Cs}{\mbox{\calligra{\scriptsize s}}\/}
\newcommand{\Ct}{\mbox{\calligra{\scriptsize t}}\/}
\newcommand{\Cu}{\mbox{\calligra{\scriptsize u}}\/}
\newcommand{\Cv}{\mbox{\calligra{\scriptsize v}}\/}
\newcommand{\Cw}{\mbox{\calligra{\scriptsize w}}\/}
\newcommand{\Cx}{\mbox{\calligra{\scriptsize x}}\/}
\newcommand{\Cy}{\mbox{\calligra{\scriptsize y}}\/}
\newcommand{\Cz}{\mbox{\calligra{\scriptsize z}}\/}
%
\newcommand{\CA}{\mbox{\calligra{\scriptsize A}}\/}
\newcommand{\CB}{\mbox{\calligra{\scriptsize B}}\/}
\newcommand{\CC}{\mbox{\calligra{\scriptsize C}}\/}
\newcommand{\CD}{\mbox{\calligra{\scriptsize D}}\/}
\newcommand{\CE}{\mbox{\calligra{\scriptsize E}}\/}
\newcommand{\CF}{\mbox{\calligra{\scriptsize F}}\/}
\newcommand{\CG}{\mbox{\calligra{\scriptsize G}}\/}
\newcommand{\CH}{\mbox{\calligra{\scriptsize H}}\/}
\newcommand{\CI}{\mbox{\calligra{\scriptsize I}}\/}
\newcommand{\CJ}{\mbox{\calligra{\scriptsize J}}\/}
\newcommand{\CK}{\mbox{\calligra{\scriptsize K}}\/}
\newcommand{\CL}{\mbox{\calligra{\scriptsize L}}\/}
\newcommand{\CM}{\mbox{\calligra{\scriptsize M}}\/}
\newcommand{\CN}{\mbox{\calligra{\scriptsize N}}\/}
\newcommand{\CO}{\mbox{\calligra{\scriptsize O}}\/}
\newcommand{\CP}{\mbox{\calligra{\scriptsize P}}\/}
\newcommand{\CQ}{\mbox{\calligra{\scriptsize Q}}\/}
\newcommand{\CR}{\mbox{\calligra{\scriptsize R}}\/}
\newcommand{\CS}{\mbox{\calligra{\scriptsize S}}\/}
\newcommand{\CT}{\mbox{\calligra{\scriptsize T}}\/}
\newcommand{\CU}{\mbox{\calligra{\scriptsize U}}\/}
\newcommand{\CV}{\mbox{\calligra{\scriptsize V}}\/}
\newcommand{\CW}{\mbox{\calligra{\scriptsize W}}\/}
\newcommand{\CX}{\mbox{\calligra{\scriptsize X}}\/}
\newcommand{\CY}{\mbox{\calligra{\scriptsize Y}}\/}
\newcommand{\CZ}{\mbox{\calligra{\scriptsize Z}}\/}
%\newcommand{\CLin}  {\mbox{\calligra{\scriptsize Lin}}\/}
%\newcommand{\CSym}  {\mbox{\calligra{\scriptsize Sym}}\/}
%\newcommand{\CSkw}  {\mbox{\calligra{\scriptsize Skw}}\/}
%\newcommand{\COrth} {\mbox{\calligra{\scriptsize Orth}}\/}
\newcommand{\CLin}  {{\mathscr Lin     }}
\newcommand{\CSym}  {{\mathscr Sym     }}
\newcommand{\CSkw}  {{\mathscr Skew    }}
\newcommand{\COrth} {{\mathscr Orth}^{+}}
%==============================================================================
% Vektoren
%==============================================================================
% Spaltenvektor mit runden Klammern
\newcommand{\svektor}[2]%
{\left(\!\!\!\begin{array}{c} #1\\#2 \end{array}\!\!\!\right)}
% Zeilenvektor mit runden Klammern
\newcommand{\zvektor}[2]%
{\left(#1,#2\right)^T}
% Spaltenvektor mit eckigen Klammern
\newcommand{\Svektor}[2]%
{\left[\!\!\!\begin{array}{l} #1\\#2 \end{array}\!\!\!\right]}
% Zeilenvektor mit eckigen Klammern
\newcommand{\Zvektor}[2]%
{\left[#1,#2\right]^T}
\newcommand{\nvec}{\bf{n}}
\newcommand{\ncmp}{\rm{n}}
\newcommand{\tvec}{\bf{t}}
\newcommand{\tcmp}{\rm{t}}
\newcommand{\xvec}{\bf{x}}
\newcommand{\xcmp}{\rm{x}}
\newcommand{\yvec}{\bf{y}}
\newcommand{\ycmp}{\rm{y}}
%==============================================================================
% Bezeichnungen fuer FEM
%==============================================================================
\newcommand{\Erzeugnis}[1]{\langle #1 \rangle}
\newcommand{\Basis    }[1]{\langle\langle #1 \rangle\rangle}
\newcommand{\Dual     }[2]{\langle #1, #2 \rangle}
\newcommand{\SProd    }[2]{( #1, #2 )}
\newcommand{\Abschluss}[2]{\overline{#1}^{\Norm{\cdot}_{#2}}}
\newcommand{\Greg}{\Ol{\Omega}}
\newcommand{\Gphi}{\phi_n}
\newcommand{\Gcoe}{U^n}
\newcommand{\Gnod}{X}
\newcommand{\Ereg}{\Ol{\Omega}_e}
\newcommand{\Elab}{{\scriptscriptstyle (e)}}
\newcommand{\Ephi}{\phi_i^\Elab}
\newcommand{\Ecoe}{\stackrel{\Elab}{U^i}}
\newcommand{\Enod}[2]{\stackrel{\Elab}{X^{#1}_{#2}}}
\newcommand{\Eass}{\stackrel{\Elab \ }{{\mbox{\large\bf A}}_n^i}}
\newcommand{\Gsum}{\sum_{n=1}^{N}}
\newcommand{\Lsum}{\sum_{i=1}^{N_e}}
\newcommand{\Esum}{\sum_{e=1}^{E}}
\newcommand{\Ecup}{\bigcup_{e=1}^{E}}
%==============================================================================
% Optische Funktionen
%==============================================================================
\newcommand{\spc    }    {\hspace{ 5mm}}              % Freiraum  5mm
\newcommand{\Spc    }    {\hspace{10mm}}              % Freiraum 10mm
\newcommand{\SPC    }    {\hspace{20mm}}              % Freiraum 20mm
%------------------------------------------------------------------------------
\newcommand     {\eg}           {\mbox{e.\,g.}\ }
\newcommand     {\bea}          {\renewcommand{\arraystretch}{1.5} \begin{array}}
\newcommand     {\eea}          {\end{array}\renewcommand{\arraystretch}{1.0} }
\newcommand     {\nnn}          {\nonumber  }
\newcommand     {\auf}          {\mbox{on}  }
%-------
\newcommand     {\intst}        {\int\!\!\!\!\!\int\limits_{\!\!\!\!\!\!\Omega   \times I}}
\newcommand     {\intbt}        {\int\!\!\!\!\!\!\int\limits_{\!\!\!\!\!\!\Gamma_N \times I}}
% 
\newcommand     {\intstn}       {\int\!\!\!\!\!\int\limits_{\!\!\!\!\!\!\Omega   \times I^n}}
\newcommand     {\intbtn}       {\int\!\!\!\!\!\!\int\limits_{\!\!\!\!\!\!\Gamma_N \times I^n}}
% 
\newcommand     {\intresth}     {\int\!\!\!\!\!\int\limits_{\!\!\!\!\!\!T_h   \times I^n}}
\newcommand     {\intrese}      {\int\!\!\!\!\!\!\int\limits_{\!\!\!\!\!\!E_h \times I^n}}
\newcommand     {\intresti}     {\int\!\!\!\!\!\!\int\limits_{\!\!\!\!\!\!T_h}}
%
\newcommand     {\glref}[1]     {(\ref{#1})}
\newcommand     {\EQ}[1]        {(\ref{#1})}
\newcommand     {\figref}[1]    {(\ref{#1})}
%------------------------------------------------------------------------------
%\newfont{\Eule}{eule scaled 2000} % Uni-Logo
%\newcommand{\logo}{{\Eule i}}
%------------------------------------------------------------------------------
\newcommand{\Fig}        [1]  {Abb.~(\ref{#1})}
\newcommand{\Eq}         [1]  {Gl.~(\ref{#1})}
\newcommand{\sdiamond}        {\mbox{\tiny$\Diamond$}}
%------------------------------------------------------------------------------
\newcommand{\DDOT}      [1]  {\stackrel{\mbox{\bf ..}}{#1}}%{\dot{#1}}
\newcommand{\DOT}       [1]  {\stackrel{\mbox{\bf  .}}{#1}}%{\dot{#1}}
\newcommand{\hgl}            {\\[+2mm]}
\newcommand{\Hgl}            {\\[+3mm]}
\newcommand{\HGl}            {\\[+4mm]}
\newcommand{\HGL}            {\\[+5mm]}
\newcommand{\hHgl}           {\\ \hline}
%------------------------------------------------------------------------------
\newcommand{\Stackrel}  [2]  {\,\stackrel{#1}{\text{$#2$}}}
